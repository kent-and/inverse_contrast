\documentclass[11pt,a4paper]{article}
\usepackage{hyperref}
\usepackage{color}
\usepackage{amsmath, amsthm, amssymb, amsfonts, verbatim}
\usepackage{graphicx}
\usepackage[table,x11names]{xcolor}
\usepackage{geometry}
\usepackage{subcaption}
\usepackage[numbers]{natbib}

\title{Assessing paravascular transport in the parenchyma by partial differential equation (PDE) constrained optimization.}

\renewcommand{\comment}[1]{\textcolor{red}{#1}}

\newcommand{\kam}[1]{\textcolor{blue}{#1}}
\newcommand{\fixme}[1]{\textcolor{orange}{#1}}
\newcommand{\lars}[1]{\textcolor{magenta}{#1}}
\author{Lars Magnus Valnes, Sebastian K. Mitusch, Geir A. Ringstad, \\ 
Per Kristian Eide, Simon W. Funke, Kent-Andre Mardal }


\begin{document}
\maketitle

\begin{abstract}
bla bla 
\end{abstract}
\section{Introduction}
In 2012, ~\citet{iliff2012paravascular} provided evidence for a a brain-wide paravascular system, denoted the glymphatic system, which is a pathway for transport of fluids and solutes. It was proposed to have an important metabolic role by providing a pathway for clearance of waste solutes from the brain. Evidence was given that the activity of the glymphatic system increases during sleep ~\citet{xie2013sleep}, thus linking sleep to clearance of toxic substances from the brain. The glymphatic system seems to become impaired during aging, and failure of the glymphatic system may play a role in neurodegenerative disease and development of Alzheimer's disease. Evidence for a brain-wide glymphatic system in humans was first obtained in 2015 \citet{eide2015mri}, and in 2017 \citet{ringstad2018brain} provided evidence that glymphatic function is deteriorated in individuals with dementia, as compared to controls.
Several aspects of the glympahtic system are still debated, for example whether glymphatic transport of solutes is by convective mechanisms or by diffusion.

%Recent breakthroughs in medicine have linked neurodegenerative diseases like Alzheimer's to insufficient sleep through a novel fluid mechanism called the glymphatic system 
%that is responsible for waste clearance. 
%In 2011,~\citet{iliff2012paravascular} discovered a novel system within our brain's metabolic system, which changed our understanding of the brain. 
%The system involves a new pathway, called the paravascular pathway as it runs in parallel with the vasculature system in a surrounding compartment.
%The important novelty in the 2011 findings was that waste from the brain is cleared through the paravascular path, and that a malfunction of this system 
%is linked to the development of dementia such as Alzheimer's and Parkinson's diseases. In 2013,~\citet{xie2013sleep} found
%that this system is particularly active during sleep.
%These mentioned breakthroughs were all performed in mice. In 2018, the action of the glymphatic system was for the first time demonstrated in humans. Further, it was shown that this action differs between healthy controls and patients with dementia~\cite{ringstad2018brain}.
%However, the biomechanical mechanisms behind the glymphatic system is not well understood from a fluid mechanics perspective,
%and so far the modeling attempts have mostly failed~\cite{asgari2016glymphatic, holter2017interstitial, smith2017glymphatic}. 
%In particular the system seems to facilitate waste transport faster than  
%the extracellular diffusion, which has been the prevailing paradigm since the pioneering works of \citet{sykova2008diffusion}.


The biomechanical mechanisms are not well understood from a fluid mechanics perspective,
and so far the modeling attempts have mostly failed~\cite{asgari2016glymphatic, holter2017interstitial, smith2017glymphatic}. 
In particular the system seems to facilitate waste transport faster than  
the extracellular diffusion, which has been the prevailing paradigm since the pioneering works of \citet{sykova2008diffusion}


%A novel pathway of the brain's metabolism system called the paravascular pathway, because it runs in parallel with 
%vasculature system in a surrounding compartment,  
%was discoved in 2011~\cite{iliff2012paravascular}. 
%%Later it was found that this system is particularly active during sleep~\cite{xie2013sleep}.  
%It has been proposed that this pathway plays an important role in the clearance of waste
%from the brain and hereby a malfunction of this system is linked to  the development of dementia such as Alzheimer's and Parkinson's diseases. That is, 
%the lymphatic system plays a crucial role in waste clearance in the rest of our body, but there are no lymph vessels inside
%the brain. As such, the metabolic process of the brain is not well understood and this is surprising since the brain is a very 
%energy demanding organ (about 10 times as demanding as the average organ). The circulatory system of the paravascular pathway 
%has therefore been named the glymphatic system~\cite{jessen2015glymphatic} as it has a similar role as the well-known lymphatic system and because the glia cells 
%are crucial in this system.  
%The glymphatic system remains controversial and modeling attempts
%has so far mostly failed to explain the biomechanics of the system~\cite{asgari2016glymphatic, holter2017interstitial, smith2017glymphatic}\kam{Lars: vi trenger flere her}. A proper biomechanical 
%understanding of the pathway has significant potential because
%dementia such as Alzheimer´s and Parkinson´s diseases are
%associated with accumulation of metabolic waste such as
%amyloid-beta and CSF-tau.  

In \cite{ringstad2018brain} brain-wide distribution of MRI-contrast was demonstrated during 24 hours after lumbar contrast injection. Brain-wide distribution by diffusion alone was deemed unlikely by the authors. The argument was based on analytical considerations where it was calculated that 50\% contrast enrichment would occur after 
55 hours using the error function which is valid for planar diffusion. However, 
the surface of the brain is folded and is around five times larger than 
a corresponding surface of a ball with the same volume. Hence, 
a more rigorous modeling attempt is warranted. A complicating factor
is however that the contrast in the surrounding cerebral spinal fluid (CSF) is heterogeneous
and changes significantly during the 24 hours of the investigations. 
Furthermore, images were obtained only at a few time-points during the investigations in \cite{ringstad2018brain}. As such, the sparseness in time prevents a direct computation of the diffusion coefficients.

The work of Sykov{\'a} and Nicholson~\cite{sykova2008diffusion} demonstrated
that diffusion was a governing transport mechanism in the brain and with an estimated order of magnitude $1.0\mathrm{e}{-4} \mathrm{mm\sp{2}/s}$ for large molecules. This is confirmed with diffusion tensor imaging (DTI) where young and healthy subjects have typical diffusion values in white matter around $0.7-0.9\mathrm{e}{-3} \mathrm{mm\sp{2}/s}$ \citet{Helenius194},
while subjects with dementia typically have higher value and variation \cite{goujon2018can} . Of course diffusion coefficients, 
measured by DTI at a time-scale of 7-30 minutes, may not be representative for the process on a
longer time-scale, at least if the paravascular pathway plays an important role. The fact that the vascular system occupies around 3\% \fixme{Eide ? } of the brain volume 
and the paravascular network is substantially smaller potentially renders the paravascular transport  invisible at the short time-scales of a DTI acquisition while it may still be effective on longer time scales. 


Our purpose in this paper is to attempt a more rigorous methodology 
for assessing the paravascular transport process where
the complex geometry of the brain as well as time-scales of hours or days are taken into account.
%That is,
Thus we aim to investigate whether we can assess an effective diffusion coefficient on long time-scales (hours or days), by fitting a diffusion model to magnetic resonance imaging (MRI) data. If the fit between observation and model is good and we identify diffusion coefficients that are 
in line with those predicted by DTI then paravascular transport can be ignored on the scale of our study. 
Our approach is to solve an optimization problem constrained by a partial differential equation (PDE) using the adjoint method where we have sparse observations on selected time-points, around 10 acquisitions during 24 hours, but through
the complete domain. Hence, the challenges faced from a mathematical point of view 
are that 1) the images are subject to noise, 2) resolution in space is limited to slightly less than 1 $\mathrm{mm}^3$ and 3)
the sparse observations through the time domain. 
Therefore we need to assess the sensitivity of the approach with respect to important factors, such noise levels and time resolution to determine whether this approach is a viable method to obtain parameters involved time-scales. 

%As such we need to assess the sensitivity of the approach with respect to important parameters such as the regularization parameters, noise levels and  time resolution to determine whether this approach is a viable method to obtain parameters involved time-scales. 
We remark that the purpose
of this paper is a systematic study of the mathematical challenges and that assessing  
whether clearance is governed by a process that is faster than diffusion is not the topic of the current paper. 

An outline of the paper is as follows. 
In Section 2 we present the methodology of the paper. Section 2.1-2.3 contains a detailed description of the medical imaging 
relevant to this study as we do not expect the reader to have prior knowledge of medical imaging. Section 2.4
describes the PDE constrained optimization problem and the corresponding solution algorithm, while 2.5 presents
a test problem using a manufactured solution that is used for method verification. Section 2.6 describes the implementation 
in FEniCS \cite{LoggMardalEtAl2012a}. In Section 3 we present the results and have a rather extensive discussion of the verification performed
using a manufactured solution (Section 3.1-3.3) before we present the results obtained by real observations in Section 3.4     
and a comparison with DTI data in Section 3.5. Finally, in Section 4, the methodology and results are discussed.  
   
\section{Methodology}

Sections 2.1-2.3 briefly describe the details of the imaging relevant for this study. More details on the MRI protocols can be
found in e.g. \cite{ringstad2018brain}. Sections 2.4-2.5 describe the solution of PDE constrained optimization problems and
its implementation in FEniCS. The exposition is, however, brief and we refer to e.g. \cite{hinze2008optimization} for more details. 



\subsection{MRI Data}
\begin{figure}
\includegraphics[width=0.95\textwidth]{PatID-68-new-100.png} 
\caption{Shows the percentage change in T1 signal unit ratios from baseline at different observation times. The colorbar was restricted to the range $(0,100)$.\fixme{copyright issues}. \lars{Reconstructed based on data obtained from \cite{eidevalnes}. eller fjerne upper row? } }
\label{fig1} 
\end{figure}

\begin{figure}
\includegraphics[width=0.95\textwidth]{Zoom-PatID-68.png} 
\caption{Shows the percentage change in T1 signal unit ratios from baseline at different observation times in the slice (marked red in the left panel) used in the subsequent analysis. The color bar was restricted to the range $(0,100)$. }
\label{fig2} 
\end{figure}
Figure~\ref{fig1} shows distribution of a CSF tracer, as a percentage change in T1 signal unit ratios, see also~\cite{ringstad2018brain} for further information on the imaging procedure.   
Our data (not all shown) consist of a total of 10 MRI observations, including a baseline MRI taken before tracer was injected. The observation points are distributed over 5 observations within 1-2 hours after injection, a single observation in the timeframes 2-4 hours, 6-9 hours, 24 hours and 48 hours. 
Figure~\ref{fig2} shows the region selected for our computations. 
The software Freesurfer \cite{Dale1999179, FischlLiuDale, spf2007, reuter:robreg10} was used to segment and align each of the observations, which made it possible to estimate voxelwise signal increase. 

%The MRI data was part of a larger study, which involves around 100 patients with the same MRI acquisitions. However, there are clinical differences in each patient, thus a thorough analysis is needed to obtain a robust and accurate method.

\subsection{Contrast concentration - Image Signal Relation}
Below, we briefly describe the relationship between the imaging signal 
seen in Figures \ref{fig1} and \ref{fig2} and the underlying contrast 
concentration. We remark that we use a notation common in medical literature and here two letter symbols are common. Hence, below we will use two letter symbols such as $TE$ and $TR$ to keep the notation consistent with the presentation in \cite{GOWLAND, MPRAGE}.   
The tracer concentration $c$ causes the longitudinal(spin-lattice) relaxation time $T_{1}$ to shorten with the following relation
\begin{equation}
\frac{1}{T_{1}\sp{c}} = \frac{1}{T_{1}\sp{0}} + r_{1}c .
\label{EQ::contrast}
\end{equation}
The superscripts indicate relaxation time with contrast $T_{1}\sp{c}$ and without contrast $T_{1}\sp{0}$, and $r_1$ is the relaxivity constant for the MRI tracer in a medium. 
The tracer observations were collected using a MRI sequence known as  Magnetization Prepared Rapid Acquisition Gradient Echo (MPRAGE) with an inversion prepared magnetization. The relation between signal and the relaxation time is non-linear, and is expressed with the following equations. The signal value $S$ for this sequence is given by
\begin{equation}
S = M_{n} \sin \theta e\sp{ - TE/T_2\sp{*} },
\label{EQ::SI_T2}
\end{equation}
with $TE$ and $\theta$ respectively denoting the echo time and the flip angle, and $M_{n}$ the magnetization for the n-echo described below. 
Also $T_2\sp{*}$ is transverse magnetization caused by a combination of spin-spin relaxation and magnetic field inhomogeneity. It is defined as 
\begin{equation}
\frac{1}{T_2\sp{*}} = \frac{1}{T_2} + \gamma \Delta B_{in} ,
\end{equation}
with $T_2$ transverse (spin-spin) relaxation time, $\gamma$ is the gyromagnetic ratio and $\Delta B_{in}$ is the magnetic field inhomogeneity across a voxel. The expression can be simplified by neglecting the $T_2$ term in the signal, since $TE <<T_2\sp{*}$ for this MRI sequence. Thus \eqref{EQ::SI_T2} becomes 
\begin{equation}
S = M_{n} \sin \theta.
\label{EQ::SI}
\end{equation}
In article \cite{GOWLAND}, the term $M_n$ is defined as the magnetization for the n-echo 
\begin{equation}
M_{n} = M_{0}  \left[ (1-\beta)\frac{(1-(\alpha \beta)\sp{n-1} }{1-\alpha\beta} + (\alpha \beta)\sp{n-1}(1-\gamma) + \gamma ( \alpha \beta)\sp{n-1} \frac{M_{e}}{M_{0}}  \right]   
\end{equation}
with 
\begin{equation}
\frac{M_{e}}{M_{0}} = - \left[ \frac{ 1 -\delta + \alpha \delta (1-\beta ) \frac{1-\alpha\beta\sp{m}}{1-\alpha \beta} + \alpha\delta(\alpha\beta)\sp{m-1} - \alpha\sp{m}\rho}{1 +\rho \alpha\sp{m} } \right].
\end{equation}
Using the following definitions
\begin{equation}
\begin{aligned}
\alpha &= \cos ( \theta ) \\
\beta  &= e\sp{- \sp{T_b}/_{T_1\sp{c}} } \\
\delta &= e\sp{- \sp{T_a}/_{T_1\sp{c}} } \\
\gamma &= e\sp{- \sp{T_w}/_{T_1\sp{c}} } \\
\rho   &= e\sp{- \sp{TR}/_{T_1\sp{c}}}  \\
T_w    &= TR - T_a -T_b(m-1)       .\\
\end{aligned}
\end{equation}
Here $T_b$ is known as the echo spacing, $T_a$ is the inversion time, $T_w$ the time delay, $TR$ as the repetition time, $m$ is the number of echo spacings and $T_1$ is the longitudinal(spin-lattice) relaxation time for a given medium. The $M_0$ is a calibration constant of the magnetization. The center echo denoted as $n=m/2$ will be the signal that we will consider when estimating MRI tracer. Given \eqref{EQ::SI} we have that the relative signal increase can be written as 
\begin{equation}
\frac{S\sp{c}}{S\sp{0}} = \frac{ M_{n}\sp{c} \sin (\theta)}{ M_{n}\sp{0} \sin (\theta) }.
\end{equation}
We define that  
\begin{equation}
f(T_1) = M_{n}/M_{0} ,
\label{scaledmagnetization}
\end{equation}
which can be seen in Figure \ref{figuredti}. 
This gives the following relation 
\begin{equation}
\frac{f(T_{1}\sp{c} ) }{f(T_{1}\sp{0})}  = \frac{S\sp{c}}{S\sp{0}} 
\end{equation}
The signal difference between observation times were adjusted in \cite{eidevalnes}. Thus we can express the change in $T_1$ due to tracer as 
\begin{equation}
f ( T_{1}\sp{c} ) = \frac{S\sp{c}}{S\sp{0}} f(T_{1}\sp{0}) 
\end{equation}
and then estimate the concentration using \eqref{EQ::contrast}. The $T_{1}\sp{0}$ values were obtained by T1-mapping of the brain using a MRI sequence known as MOLLI5(3)3 \cite{TAYLOR201667}. This takes into account patients specific characteristic, such as tissue damage. Tissue damage can be observed in the MRI due to a lower signal in the white matter compared to healthy white matter tissue, thus damaged tissue have different $T_1$ relaxation time. 
The tracer concentration was estimated in a preprocessing step, using the parameters obtained from the T1-map, MPRAGE MRI protocol \cite{eidevalnes} and the value for $r_1$ found in \cite{pmid16230904}. In the computation, the function \eqref{scaledmagnetization} was computed for $ T_1\in \lbrace 200, 4000 \rbrace$ creating a lookup table. The lookup table was utilized with the baseline signal increase to estimate $T_1\sp{c}$, and then the concentration was computed using \eqref{EQ::contrast}.  

\begin{figure}
\centering
\includegraphics[width=0.70\textwidth]{T1function.png} 
\caption{Shows the  function defined in \eqref{scaledmagnetization} where the white region indicates $T_1$ values for white matter, the grey region indicates $T_1$ values for grey matter, the blue region indicates $T_1$ values for CSF.  }
\label{figureF} 
\end{figure}


\subsection{Diffusion tensor imaging}
In addition to the T1 and T2 weighted sequences described above we have also obtained diffusion tensor imaging to assess the apparent diffusion 
coefficients on short time-scales. Images are shown in Figure~\ref{figuredti} where the largest diffusion coefficient (shown in 
red in the middle figure) is shown to be around $1.0\mathrm{e}{-3}  \mathrm{mm^2/s}$. We remark that we have not included possible anisotropy, shown in 
the right-most image in Figure~\ref{figuredti} and that these images show the apparent diffusion coefficient for free water molecules (18 Da).      
The diffusivity of the Gadovist \cite{MGadobutrol} was estimated to be similar to the diffusion coefficient of Gd-DPTA (550 Da \cite{MGgDPTA}). This is due to the fact that both molecules have similar mass, and based on Stoke-Einstein equation should also have similar diffusion coefficients. The free diffusion coefficient for Gd-DPTA was estimated in \citet{GdDPTA-DIFFUSION} to be $3.8\mathrm{e}{-4}\mathrm{mm\sp{2}/s}$.
The fractional anisotropy is defined as 
\begin{equation}
FA\sp{2} =  \frac{3}{2} \frac{ (\lambda_1 - MD )\sp{2} +(\lambda_2 - MD )\sp{2} +(\lambda_3 - MD )\sp{2}}{\lambda\sp{2}_1 + \lambda\sp{2}_2  +\lambda\sp{2}_3 },
\end{equation}
with the mean diffusivity $MD$ defined as 
\begin{equation}
MD = \frac{\lambda_1 +\lambda_2 +\lambda_3 }{3}.
\end{equation}
In these equations $\lambda_i$ denotes the eigenvalues of the diffusion tensor.
\begin{figure}
\centering
\includegraphics[width=0.80\textwidth]{DTI-zoom.png} 
\caption{The left panel show the anatomical map. The middle panel show the apparent diffusion coefficients (ADC) obtained from DTI. The right panel shows the computed fractional anisotropy (FA) from the DTI.}
\label{figuredti} 
\end{figure}


\subsection{Mathematical Model}
In \citet{sykova2008diffusion}, it was shown that the macroscopic diffusion in the brain can be considered a hindered diffusion with an apparent diffusion coefficient (ADC). The relation between the diffusion coefficients were defined as 
\begin{equation}
 \lambda =  \sqrt {D/D_{ADC}}
\label{tortuosity}
\end{equation}
with $\lambda$ denoted as the tortuosity.
In order to estimate the apparent diffusion coefficient involved in the tracer transportation shown in Figure~\ref{fig1} we assume that the process can be modeled by a diffusion equation. 
Then we constructed an optimization problem with the aim to minimize the difference between the observed and the modeled tracer distribution by optimizing the boundary conditions and the apparent diffusion coefficient.%Then we solve a PDE constrained optimization problem where the tracer distribution, boundary conditions, and apparent diffusion coefficient are solved for by optimizing with respect to the observed tracer distribution \lars{Ikke helt oversiktlig}. 
Thus enhanced transportation because of effects such as dissipation would result in an apparent diffusion coefficient larger than that predicted by DTI.The objective function was defined as 
\begin{equation}
\min_{D,g} \quad \sum\limits_{i=1}\sp{n} \int\limits_{\Omega} |u(t_i) - u_{obs}(t_i)|\sp{2} \mathrm{d}\Omega + \int\limits_{0}\sp{T} \int\limits_{\partial \Omega_1} \left( \frac{\alpha}{2} | g |\sp{2} + \frac{\beta}{2} \left| \frac{\partial g}{\partial t} \right|\sp{2} \right) \mathrm{d}\Omega \mathrm{d}t  
\label{EQ::objf}
\end{equation}
subject to   
\begin{equation}
\begin{aligned}
\frac{\partial u}{\partial t} &= \nabla \cdot  D \nabla u && \text{in} \qquad \Omega \times \left\lbrace 0 , T \right)  \\
u&=g && \text{on} \qquad \partial\Omega  \times \left\lbrace 0 , T \right) 
\end{aligned}
\label{Eq::PDE}
\end{equation}
Here, $u$ is the contrast distribution, $D$ is the apparent diffusion 
coefficient, $g$ is the boundary condition, $\Omega$ is the domain, and $T$ is the final simulation time. We assume that the domain $\Omega$ consists of three sub domains, each with a different diffusion coefficient. We denote the CSF (subarachnoid and lateral ventricle) domain as $\Omega_1=\Omega_{CSF}$, the grey matter as $\Omega_{GM}$ and the white matter as $\Omega_{WM}$. 
The apparent
diffusion constant is assumed to be constant within the CSF, grey and 
white matter but each region may have different values.  
The Dirichlet boundary condition is only applied on the outer facing boundary of the CSF domain, $\partial \Omega_1$. Homogeneous Neumann conditions are applied on the remaining boundaries.
The $\alpha$ and $\beta$ parameters are non-negative regularization parameters 
and $u_{obs}$ are the observations at time-points $t_i$. 

The mesh construction was patient-specific and used the MRI of a patient diagnosed with idiopathic normal pressure hydrocephalus (iNPH). The software Freesurfer was used to segment and create the polyhedral surfaces of the white and grey matter. Then the T2 weighted MRI \cite{eidevalnes} was used to segment the CSF compartment surrounding the cerebral and the lateral ventricles. CGAL \cite{cgal:rty-m3-18b} was used to combine the polyhedral surfaces and to construct the mesh. The computational requirement for the resulting mesh was significant, therefore two submeshes was also constructed, see Fig.\ref{Fig::Mesh}.
\begin{figure}
\centering
\includegraphics[scale=0.2]{mesh.png} 
\caption{The leftmost image A) shows the mesh created from the baseline MR image with 3 domains, while the rightmost image  B) shows the mesh created from the baseline MR image with 2 domains. The blue domain corresponds to CSF domain, $\Omega_{CSF}$, the purple domain corresponds to grey matter,  $\Omega_{GM}$, and the white domain corresponds to white matter, $\Omega_{WM}$. }
\label{Fig::Mesh}
\end{figure}
The 3 domain mesh in Figure~\ref{Fig::Mesh} A, consists of 244318 tetrahedral cells and 22057 vertices, while the 2 domain mesh (without the CSF compartment) Figure~\ref{Fig::Mesh} B consists of 335589 tetrahedral cells and 73002 vertices. 
%The mesh coordinates (X,Y,Z) are given in $\mathrm{mm}$ \lars{ with $X \in \lbrace -70.5,-18\rbrace, Y \in \lbrace -73.5,28.8  \rbrace ,Z \in \lbrace 16.6,60.7 \rbrace$ }.


\subsection{Manufactured Solution}
In order to asses the robustness and accuracy of our strategy and test the dependency of the computed diffusion coefficients $D$ on the
regularization parameters $\alpha$ and $\beta$, we perform a test case with a
known solution.  
The manufactured observations were obtained by forward computation of \eqref{Eq::PDE} with the Dirichlet boundary condition defined as
\begin{equation}
g(t) = 0.3 +0.167t - 0.007t\sp{2} \qquad \text{ for } 0 \leq t \leq 24.
\label{EQ::DIRI}
\end{equation}
The initial condition was set to 0 everywhere, the timestep was $dt = 0.2$, and the diffusion coefficients were selected to be 
%\fixme{describe discretisation in time and space}\lars{ cell-radius ratio ? angle ?}
\begin{equation}
D_{\Omega_1} = 1000.0, \quad D_{\Omega_2} = 4.0, \quad D_{\Omega_3} = 8.0 
\end{equation}  
The magnitude order for $D_{\Omega_1} $ and $D_{\Omega_2}$ were chosen to resemble diffusion coefficient for water in grey and white matter, but the relation between the coefficients were not preserved. In \citet{Haga}, the drug concentration moved 1 cm per 20 seconds in the cervical region, thus $D_{\Omega_3}$  was set to 1000. The manufactured forward computation produced a total of 120 possible observation times.


\section{Implementation}
%In the acquired data, the MPRAGE had a high signal to noise ratio and the T1 map had low signal in the CSF. This made it difficult to estimate the concentration in the CSF compartment. Thus an additional mesh without the CSF compartment was constructed, see Fig. \ref{Fig::Mesh}.     

The solver for \eqref{Eq::PDE} was implemented using the FEniCS project (v.2017.2), using backwards Euler time discretization and first order continuous Galerkin finite elements in space. The resulting linear problems were solved using GMRES.
The module dolfin-adjoint \cite{farrell2013automated, funke2013framework} was used to automatically derive and solve the adjoint problem, and to solve the PDE constrained optimization problem with the L-BFGS-B algorithm \cite{LBFGSB1, LBFGSB2}. The optimization was stopped when the $L^\infty$-norm of the projected gradient of the objective functional dropped below $1.0\mathrm{e}{-1}$.

The observations $u_{obs}$ have fixed time-points. However, these time-points may not coincide  with the time discretization in the forward problem $t_j$. Therefore, we linearly interpolated the computed solutions onto the observation time points using the formula
\begin{equation}
\label{observation:interpolation}
u(t) \approx \frac{\Delta t}{dt} u(t_{j-1}) + \frac{dt - \Delta t }{dt} u(t_{j}), \quad t \in \lbrace t_{j-1}, t_j \rbrace
\end{equation}
with $\Delta t = t_j-t$ and $dt = t_j - t_{j-1}$.

The implementation used the first observation as initial conditions of \eqref{Eq::PDE}. Then for each time-step, the next observation was used as the initial value for boundary control $g$. In order to increase the convergence speed of the optimization, $D_{\Omega_1}$ was scaled so that $D_{\Omega_1} = 100 D_{\Omega_1}\sp{\star} $. The initial values of the diffusion coefficients in the optimization algorithm were $(D_{\Omega_1}\sp{\star}, D_{\Omega_2}, D_{\Omega_3})=$  $(1, 1, 1)$. 

The noise susceptibility was tested by pointwise adding a uniform distributed noise to the observations. The noise term was constructed using the random library in numpy and adjusted so that the noise range was $\lbrace -n_{amp} , n_{amp} \rbrace $, with $n_{amp}$ denoted as noise amplitude.


\section{Results}

\subsection{Verification in terms of the manufactured solution}
Below we will discuss the parameter identification and its sensitivity with respect to the regularization parameters, noise, number of observations and time-resolution of the forward model. When presenting the results, we denote the relative error of the diffusion parameters as 
\begin{equation}
 D_{\Omega_i}\sp{rel} = \frac{D_{\Omega_i}\sp{found} -D_{\Omega_i}\sp{true} }{D_{\Omega_i}\sp{true} }, \text{ for } i=1,2,3
\end{equation}
For the boundary parameter $g$, the relative error norm is defined as 
\begin{equation}
|| g ||\sp{rel} = \frac{ \sum\limits_{t_j} || g_{found}(t_j) -g_{true}(t_j)||_{L^2(\Omega_1)} }{  \sum\limits_{t_j}||g_{true}(t_j)||_{L^2(\Omega_1)} }
\end{equation}
The case when the minimization algorithm fails to converge within 1000 iteration will be indicated by a hyphen in Table~\ref{TAB::timesteps}, Table~\ref{TAB::double} and Table~\ref{Tab::Noise0.3}

\subsubsection{The relaxation parameters}

The convergence as well as the result of the minimization is influenced by the values of the regularization parameters $\alpha$ and $\beta$. Therefore the convergence was examined by a systematic study of the reconstruction of the manufactured solution with respect to a wide range of regularization parameters. 

%To assess the sensitivity of the the objective functional defined in Eq.\ref{Eq::F} with respect to the regularization parameters $\alpha$ and $\beta$ 
%we perform a systematic study of the reconstruction of the manufactured solution with respect to a wide range of regularization parameters. 

Ideally, our approach should have a wide range of parameters in which the reconstruction algorithm yields very similar end-results although the convergence may vary substantially. The evaluation of different regularization parameters was done by solving the optimization with different values of $\alpha$ and $\beta$ and comparing the computed $D$ and $g$ values with the manufactured solution. The observation times were set to $\lbrace$ 2.4, 4.8, 7.2, 9.6, 12.0, 14.4, 16.8, 19.2, 21.6, 24.0 $\rbrace$, and the number of time-steps $k$ in the forward problem was set to $10$. 

The results are shown in Table~\ref{Tab::1} and the convergence is shown in Figure~\ref{convergence}. We can see that the optimization computed similar end-result for most regularization parameters. More specifically, for $\alpha \leq 1\mathrm{e}{-2}$ and $\beta \leq 1e-0$ the relative errors for all optimization parameters are below $5\%$.However, in the case of $ \alpha=1.0$ the relative error is over 100\% higher then for other values of $\alpha$. Furthermore, for all $\beta=100.0$ there is about 70\% relative error for $D_{\Omega_1}$ and approximate 10\% error for $D_{\Omega_2}$ .


From the results it can be discerned that $\alpha =1.0$ and $\beta=100.0$ represent an upper threshold where the observations are no longer the dominant term in the objective functional.  


\begin{figure}
\centering
\includegraphics[scale=0.18]{Convergence-plot.png}   
\caption{Convergence plots of the diffusion coefficients, boundary conditions and functional with respect to different $\alpha$ and $\beta$ values. } 
\label{convergence}
\end{figure}

\subsubsection{The number of timessteps}
Increasing the number of time-steps in the computations can potentially allow for a more accurate temporal reconstruction, but at the same time the number of observations per control decreases and the regularization may therefore play a larger role. Furthermore, the computational cost will rise with additional time-steps. Thus finding the optimal number of time-steps can be essential for solving larger problems. The computation used a wide range the regularization parameters, and the number of timesteps were selected to be 20 and 40.

The results are given in Table~\ref{TAB::timesteps}. It can be seen for all $\alpha=1.0$ that the relative error becomes bigger with more time-steps. This behavior can also be seen for $\alpha=1.0\mathrm{e}{-2}$ and $\beta < 1.0\mathrm{e}{-2}$, which implies an interaction between the regularization parameters. Therefore $g$ was investigated by plotting boundary points through time for different values of $\alpha$ and $\beta$, see Fig \ref{boundarycontrol}. In Figure \ref{boundarycontrol} we can see that more time-steps causes oscillations for $\alpha=1.0$, and each peak corresponds to a observation time. These oscillations are caused by the value selection of $\alpha$, which tries minimize $g$ between observation times.

%It can be seen that more time-steps causes an increase number of iterations, but for some values the number of iterations decreases. These values share that $\alpha $ is a magnitude 2 lower than $ \beta$,i.e $\alpha=1.0e-2$ and $\beta =1.0$. This indicates that $\alpha $ should be smaller than $\beta$ to achieve optimal convergence. However, the number of iterations in Table ~\ref{Tab::1} are smaller for all regularization parameters. 


\subsubsection{The number of observations}
The dependency on observations is relevant, since the number of observations of the MRI data is limited. This dependency was investigated by changing the number of observations, and by examining the results. The number of observation were chosen to be 5 and 20, and the observation times were evenly spaced. 


The results for 5 observations is shown in Table~\ref{TAB::half} and for 20 observations is shown in Table~\ref{TAB::double}. In Table~\ref{TAB::half}, the values $\alpha =1.0\mathrm{e}{-2}$ and $\beta<1.0$ gives a surge in the relative error with more time-steps. This behavior is neither visible in Tab.\ref{TAB::timesteps} or in  Tab.\ref{TAB::double} for the same regularization parameters. The ratio observations per control indicates that the increase in relative error is due to oscillations, similar to those seen in Figure~\ref{boundarycontrol}. Thus the observations acts as stabilization for the functional, which gives a broader range of adequate regularization parameters.

%In Table ~\ref{TAB::double}, we can see that $\aplha < 1.0e-2$ and $\beta =1.0e-2$ have a drastic change in the number of iteration for $20$ time-steps. 




\begin{figure}
\centering
\includegraphics[scale=0.21]{boundary_control.png}  
\caption{ Displays plots over time for a selection of points at the boundary of $\Omega_1$ with different regularization parameters and number of timesteps $k$. The left panel shows the legend for the plot over time, together with the selection of points. The middle panel shows the average boundary value $g$ for different regularization parameter with $k=10$. The right panel shows the average boundary value $g$ for different regularization parameter with $k=40$.  }
\label{boundarycontrol}
\end{figure}

\subsubsection{The noise susceptibility}
The MRI data contains noise, hence an investigation to the noise susceptibility is needed. This was done by pointwise adding uniform distributed noise in the range $\lbrace -n_{amp} , n_{amp} \rbrace $, with $n_{amp}$ defined as noise amplitude, to the manufactured observations. Then the optimization was solved by varying the regularization parameters $\alpha$ and $\beta$. Figure~\ref{12hourswithnoise} and Figure~\ref{24hourswithnoise} show the reconstruction of the manufactured solution at
different noise levels with $\alpha=0.0001$,$\beta=1.0$ and $k=20$. Clearly, the reconstruction shown in the lower rows is quite robust with respect to the various levels
of noise shown in the upper rows.      

In Table~\ref{Tab::Noise0.03}, we can see that a noise amplitude of 0.03, (10$\%$ of maximum initial condition) had negligible effect on the relative errors compared zero noise in Table~\ref{Tab::1} and Table~\ref{TAB::timesteps}. Then in Table~\ref{Tab::Noise0.3} the noise amplitude was increased to 0.3 (100$\%$ of maximum initial condition), which gave a visible effect. The noise caused the number of iteration to increase, and with more timesteps caused the optimization not to convergence for $\beta < 1.0$. Furthermore, the relative boundary error $||g||\sp{rel}$ is significantly larger for $\alpha< 1.0\mathrm{e}{-2} $ and $\beta < 1.0\mathrm{e}{-2}$.  




\begin{figure}
\centering
\includegraphics[scale=0.4]{noise-12.png}  
\caption{The upper row shows the manufactured observation, A ) Shows the manufactured observation at time-point 24 with no noise added. B) Shows the manufactured observation at time-point 24 with an noise amplitude of 0.15. C)Shows the manufactures observation at time-point 24 with noise amplitude of 0.3. The lower row shows the results with optimized parameter obtained with $\alpha=0.0001$, $\beta=1.0$ and $k=20$. D) Shows the resulting state given the observation in A. E)  Shows the resulting state given the observation in B .F) Shows the resulting state given the observation in C. }
\label{12hourswithnoise}
\end{figure}


\begin{figure}
\centering
\includegraphics[scale=0.4]{noise-24.png}
\caption{The upper row shows the manufactured observation, A ) Shows the manufactured observation at time-point 24 with no noise added. B) Shows the manufactured observation at time-point 24 with a noise amplitude of 0.15. C)Shows the manufactures observation at time-point 24 with noise amplitude of 0.3. The lower row shows the results with optimized parameter obtained with $\alpha=0.0001$, $\beta=1.0$ and $k=20$. D) Shows the resulting state given the observation in A. E)  Shows the resulting state given the observation in B .F) Shows the resulting state given the observation in C.  }
\label{24hourswithnoise}
\end{figure}




%\subsection{Sparse observations}
%The MRI data contains observations after the injection of CSF tracer. These observation are unevenly distributed in time resulting in large time intervals with no observations. Therefore the effect of these temporal gaps in the observations will be evaluated. This was done by selecting the observation times as follows  $t_i = \lbrace$0.8, 1.0, 1.2, 1.8, 2.4, 3.6, 5.4, 7.6, 24.0$\rbrace$. The regularization parameters was selected, so that the error would be minimized,and noise was also added.
%
%The results are shown in Tab.\ref{Tab::Hole1} and Tab\ref{Tab::Hole2}, and it can be observed that the addition of noise caused $\beta=1.0$ to converge the same. Comparing the values in Tab\ref{Tab::Hole1} and Tab.\ref{TAB::timesteps}, showed that the relative error increased.  

\subsection{A Real case} 

\subsubsection{Results obtained when using real observations}
The MRI data consisted of the observations at times $t_i \in \lbrace$ 0.00, 0.16, 0.39, 0.55, 0.77, 2.09, 6.05, 24., 48., 698. $\rbrace \mathrm{(hours)}$ with the time $t_i=0.00$ as the observation 1-2 hours after the tracer was injected. There was no significant visible change in the tracer between the observations at $\lbrace $ 0.00, 0.16, 0.39, 0.55, 0.77 $ \rbrace$. This prompted the use of the following observation times  $0.00, 2.09, 6.05, 24., 48.$. 

The estimation of tracer concentration proved difficult in the CSF compartment. Therefore the 2 domain mesh, shown in Fig.\ref{Fig::Mesh} was used in the computation. The boundary control was set to the external boundary of both domains, and the subscript in $D_{\Omega_{GM}}$ and $D_{\Omega_{WM}}$ denotes grey and white matter diffusion coefficients. Furthermore, bounds were added to the L-BFGS-B algorithm to ensure non-negative boundary controls and the convergence criteria was adjusted so that the optimization was stopped when the $L^\infty$-norm of the projected gradient of the objective functional dropped below $6.0\mathrm{e}{-1}$. The number of timesteps were selected to be $k=24$ and $k=48$, which gives $dt = 2 \mathrm{h}$ and $dt = 1 \mathrm{h}$.   

The results are shown in Tab.\ref{Tab::Real-data}, and the observations after 12 and 48 hours were compared with the corresponding states in Figure~\ref{Fig::realdata}. From Table~\ref{Tab::Real-data},it can be observed that $D_{\Omega_{WM}}$ has 24\% higher value on average for $\alpha =1.0\mathrm{e}{-2}$. It can also be observed that the $D_{\Omega_{GM}}$ have a consistent increases with a decreasing $\beta$.

\subsubsection{Comparison with results obtained from DTI analysis}
 
The median diffusion coefficient in the DTI was estimated to be $8.7\mathrm{e}{-4} \mathrm{mm\sp{2}/s}$ in white matter and $1.0\mathrm{e}{-3} \mathrm{mm\sp{2}/s}$ in grey matter. This corresponds to a tortuosity of $1.85$ and $1.73$, given that the self-diffusivity of water has been estimated to be around $3.0\mathrm{e}{-3}\mathrm{mm\sp{2}/s}$ at $37\sp{o}C$. Then the reference value $3.8\mathrm{e}{-4} \mathrm{mm\sp{2}/s}$ for Gd-DPTA gives an estimate for the apparent diffusion coefficient in the grey and white matter to  be $ 1.26\mathrm{e}{-4}\mathrm{mm\sp{2}/s}$ and $1.10\mathrm{e}{-4} \mathrm{mm\sp{2}/s}$ respectively. This estimation assumes that the tortuosity is independent for molecules with mass lower than 1kDa. 

The oscillation in Fig.\ref{boundarycontrol} showed a large discrepancy in the states between observations. Therefore the states with $k=48$ were examined after 36 hours, see Figure\ref{statecomparison}. It can be observed that the upper right row displays a discrepancy, which likely due to inadequate regularization parameters. Furthermore, we can see that for $\alpha=1.0\mathrm{e}{-2}$ there is a clear discrepancy in the diffusion coefficients compared to other values $\alpha$. Excluding the values with $\alpha=1.0\mathrm{e}{-2}$ gives an average of $ 0.65 \mathrm{mm\sp{2}/h}$ in grey matter and $ 0.8 \mathrm{mm\sp{2}/h}$ in and white matter. Scaled to $\mathrm{mm\sp{2}/s}$ gives the corresponding values $1.8\mathrm{e}{-4}\mathrm{mm\sp{2}/s}$ and $2.22\mathrm{e}{-4} \mathrm{mm\sp{2}/s}$. This gives a difference of 42\% in grey matter and $ 100 \%$ in white matter compared to the estimated values using DTI.

%\subsection*{Compare high DTI values in white matter with computed}
%\fixme{Can we remove this subsection header?}

In Fig.\ref{Fig::realdata}, it can be seen that there were regions with negligible amount of tracers and vice versa. The regions with above average amount of tracers corresponds with high diffusivity regions in Figure~\ref{figuredti}. This suggests that the computed diffusion coefficients corresponds to these values. The ADC value in the high diffusivity regions are approximately  $1.4\mathrm{e}{-3} \mathrm{mm\sp{2}/s}$, which corresponds to estimated diffusion coefficient of value $1.5\mathrm{e}{-3} \mathrm{mm\sp{2}/s}$. This gives a difference of $ 46 \%$ compared to the computed values.


















%The region with high diffusivity in the white matter, shown in Fig\ref{FIG::DTI}, gives a upper bound on the diffusion coefficient to be 1.3 $\mathrm{mm\sp{2}/s}$. 



%
% \begin{table}
%\centering
%\caption{Viser omgjøringer,estiamering og prosentvis forskjell for min og max control/optimerte verdier . Unit $\left[ \mathrm{mm\sp{2}/h} \right]$}
%\resizebox{\textwidth}{!}{\begin{tabular}{*{6}c}
%$ D\sp{free}_{water} $ & $ D\sp{DTI}__{g} $ & $ D\sp{DTI}_{w} $ & $D\sp{fre}__{Gd-DPTA} $ & $D\sp{ADC}__{g} $ & $ D\sp{ADC}_{w} $ & $ D\sp{ADC}_{g} -D\sp{OPT}_{g} $ &$ D\sp{ADC}_{w} -D\sp{OPT}_{w} $ \\
%\hline
%10.8 & 3.64 & 3.13 & 1.37 & 0.46 & 0.40 &  20 - 114 $\%$  & 100- 200 $\%$
%  -   &  -    & 4.86 &  - &  -   & 0.61 &   -                  &   0 - 100 $\%$       
%\end{tabular}} 
%\label{}
%\end{table} 
% 
%The region with high diffusivity in the white matter, shown in Fig\ref{FIG::DTI}, gives a upper bound on the diffusion coefficient to be 1.3 $\mathrm{mm\sp{2}/s}$

\section{Discussion}
The methodology presented here for identification of diffusion coefficients and boundary conditions with application to the glymphatic system appears to work quite 
well for regularization parameters varying by orders of magnitude. The regularization parameters  $\alpha \in \lbrace 1.0\mathrm{e-6}, 1.0\mathrm{e}{-2} \rbrace$, $\beta \in \lbrace 1.0\mathrm{e}{-2} , 1.0 \rbrace$ with the requirement $\alpha / \beta < 1.0\mathrm{e}{-2}$ gave a relative error of  
4.1\% in grey matter and 3.6 \% in white matter. It is particularly interesting to see that the procedure efficiently removes noise as demonstrated
in the Figures \ref{12hourswithnoise} and \ref{24hourswithnoise} with noise amplitude of 23\% of maximum value. However, the addition of noise required that $\beta = 1.0 $ to obtain consistent convergence. 
Crucial in our application is the interplay between the regularization term that regulates the smoothness of the boundary conditions as well as the integrated magnitude of the boundary conditions over time, i.e. $\alpha$ and $\beta$. Further, we observed that the importance of interplay of these parameters increases with the number of time steps. This is, however, not surprising because our observations are sparse
in time and hence an oscillating boundary condition in time will minimize the integrated boundary condition at the cost of reducing the smoothness in time. The selected range of regularization parameters showed a decrease in error with more timesteps, rather than oscillation.     

While a more comprehensive study involving more patients would be required in order to assess clearance in health and disease, a few remarks here are in order. 
First of all, the process investigated here is the clearance of Gadovist from CSF into the interstitium which may differ from the clearance of metabolic waste 
from the interstitium via the CSF. That said, 
we find that the diffusion coefficients is 42-100\% larger than what was found by the DTI modality. As such, we predict that the paravascular system 
plays a significant role.  
However, we have not yet been able to assess the self-diffusion of water as well as the diffusivity of Gadovist used in this study with phantom models. As such 
the values used with \eqref{tortuosity} must be taken with caution. Furthermore,    
the computational model assumes isotropic diffusivity, but the anisotropy in the white matter is well documented, as shown by the FA in Figure~\ref{figuredti}. It can be seen in Figure~\ref{Fig::realdata} that the region with high FA have negligible amounts of tracer present. Therefore it would seem that the anisotropy do not have direct impact on the computations, since the computation can not evaluate a static environment. Thus observations of tracer in regions with anisotropy can be considered a requirement before adding anisotropy to the model.

The computational model uses 2 global controls for the diffusion coefficients, while it can be seen in  Figure~\ref{figuredti} that diffusion coefficients can be considered a spacial function. However, the implementation of control parameters for each degree of freedom would significantly increase the computational cost. Since the diffusivity appears to be regional, using region specific control parameters seems to be a better option.
It can also be taken into consideration to model the diffusion coefficients as a function in time, given the report of increased clearance in rodents during sleep \cite{xie2013sleep}. This would indicate a time-dependent diffusivity, given that the clearance in the brain is driven by diffusion. %The time-dependency would correspond to a change in tortuosity, since properties of the tracer molecule do not change.
The hypothesis of the glymphatic system \cite{iliff2012paravascular} states that the waste is cleared through the veins in the parenchyma. This gives the tracer additional pathways that is not considered in the computational model. These additional pathways can be modeled as a drainage, which can be included as a control source term. 


In Tab.\ref{Tab::Real-data}, it was shown that the grey matter diffusion coefficient had a negative correlation with the relaxation parameter $\beta$. This was caused by the boundary control $g$, which exists along the entire boundary of the grey matter. Considering that the grey matter volume has an average thickness less than 3 mm makes it susceptible to changes in $g$. This can be observed in the reconstruction in Fig.\ref{12hourswithnoise} and Fig.\ref{24hourswithnoise}, where the noise is present at the boundary.
  
Previous studies~\cite{holter2017interstitial, smith2017glymphatic} suggest that diffusion dominates in the interstitium. Furthermore, 
~\cite{asgari2016glymphatic, brynjfm, Diem} have found that dissipation in the paravascular spaces adds less than a factor two
to diffusion for solute transportation. If we take into account then that the vasculature occupy 3-5\% of the volume
of the brain and the paravascular space presumably occupy less space, then the factor 30\% (?) that we find here 
seems significant and it is difficult to explain or ignore.      

In summary, using this model, we estimated diffusion coefficients that were 43-100\% larger than estimated by DTI acquisitions. Our results indicate that dissipation of solutes along the paravascular rout has a factor of 30, as compared to diffusion alone. However, the present study did not examine whether transport is by convective mechanisms.


 
%\section{Conclusion}

%It can be concluded from the results that increasing the number of iterations gives rise to oscillations for poor regularizations parameters. 


%\begin{figure}
%\centering
%\includegraphics[scale=0.4]{27-12-hours-scale-0-1-3.png}  
%\caption{The upper row shows the manufactured observation, A ) Shows the manufactured observation at time-point 24 with no noise added. B) Shows the manufactured observation at time-point 24 with an noise amplitude of 0.15. C)Shows the manufactures observation at time-point 24 with noise amplitude of 0.3. The lower row shows the results with optimized parameter obtained with $alpha=0.0001$, $\beta=1.0$ and $k=27$. D) Shows the resulting state given the observation in A. E)  Shows the resulting state given the observation in B .F) Shows the resulting state given the observation in C. }
%\end{figure}
%
%
%\begin{figure}
%\centering
%\includegraphics[scale=0.4]{27-24-hours-scale-0-1.png}
%\caption{The upper row shows the manufactured observation, A ) Shows the manufactured observation at time-point 24 with no noise added. B) Shows the manufactured observation at time-point 24 with a noise amplitude of 0.15 {\color{red} explain noise ampliutde}. C)Shows the manufactures observation at time-point 24 with noise amplitude of 0.3. The lower row shows the results with optimized parameter obtained with $alpha=0.0001$, $\beta=1.0$ and $k=27$. D) Shows the resulting state given the observation in A. E)  Shows the resulting state given the observation in B .F) Shows the resulting state given the observation in C.  }
%\end{figure}
% 
% 
% 
%
% 

\subsection{Acknowledgements}
The computational experiments were performed on the Abel Cluster, owned by the University of Oslo and Uninett\textbackslash Sigma2, and operated by the Department for Research Computing at USIT,the University of Oslo IT-department. \url{http://www.hpc.uio.no/} 



%\bibliographystyle{plainnat}
\bibliographystyle{abbrvnat}
\bibliography{references}

 


\clearpage
 
\begin{figure}
\centering
\includegraphics[width=0.95\textwidth]{different.png} 
\caption{Row A) shows the observation at times 2 hours, 6 hours, 24 hours and 48 hours after the first observation with tracer. Row B) shows the corresponding states with the relaxation parameters $\alpha=0.01$ and $\beta=0.01$ and $k=48$.   Row C) shows the corresponding states with the relaxation parameters $\alpha=1.0\mathrm{e-4}$ and $\beta=1.0$ and $k=48$
 Row D) shows the corresponding states with the relaxation parameters $\alpha=1.0\mathrm{e-6}$ and $\beta=0.1$ and $k=48$. The color-bar was restricted to the range $ \lbrace 0 ,0.5 \rbrace$. }
\label{Fig::realdata}
\end{figure}

\begin{figure}
\centering
\includegraphics[width=0.95\textwidth]{Statecomparison36h-pinta.png} 
\caption{ Displays the states for different regularization parameters after 36 hours for $k=48$.}
\label{statecomparison}
\end{figure}

 


\begin{table}
\centering
\caption{Shows the regularization parameters $\alpha$ and $\beta$, number of timesteps $k$ with the resulting number of optimization iterations and the relative error for the control parameters. The observations times were chosen $ t_i=[2.4, 4.8,$ 7.2, 9.6, 12.0, 14.4, $16.8, 19.2, 21.6, 24.0]$ }
\begin{tabular}{*{8}c}
$\alpha$ & $\beta$ & k  & iter & $ D_{\Omega_1}\sp{rel}$ & $ D_{\Omega_2}\sp{rel} $ & $D_{\Omega_3}\sp{rel}$ & $||g||^{rel}$ \\
\hline 
 1.0e+00 	 & 1.0e+02 	 & 10 & 437 	 & +4.631 & +0.283 & +0.103 & +0.127 \\ 
 1.0e+00 	 & 1.0e+00 	 & 10 & 129 	 & +1.638 & +0.224 & +0.089 & +0.119 \\ 
 1.0e+00 	 & 1.0e-02 	 & 10 & 129 	 & +1.602 & +0.223 & +0.089 & +0.119 \\ 
 1.0e+00 	 & 1.0e-04 	 & 10 & 151 	 & +1.608 & +0.223 & +0.089 & +0.119 \\ 
 1.0e+00 	 & 1.0e-06 	 & 10 & 133 	 & +1.603 & +0.223 & +0.089 & +0.119 \\ 
 \hline
 1.0e-02 	 & 1.0e+02 	 & 10 & 186 	 & +0.649 & +0.111 & +0.036 & +0.033 \\ 
 1.0e-02 	 & 1.0e+00 	 & 10 & 125 	 & -0.027 & +0.041 & +0.036 & +0.026 \\ 
 1.0e-02 	 & 1.0e-02 	 & 10 & 133 	 & -0.039 & +0.039 & +0.036 & +0.026 \\ 
 1.0e-02 	 & 1.0e-04 	 & 10 & 108 	 & -0.039 & +0.039 & +0.036 & +0.026 \\ 
 1.0e-02 	 & 1.0e-06 	 & 10 & 154 	 & -0.039 & +0.039 & +0.036 & +0.026 \\ 
 \hline
 1.0e-04 	 & 1.0e+02 	 & 10 & 212 	 & +0.636 & +0.108 & +0.035 & +0.032 \\ 
 1.0e-04 	 & 1.0e+00 	 & 10 & 143 	 & -0.036 & +0.039 & +0.035 & +0.026 \\ 
 1.0e-04 	 & 1.0e-02 	 & 10 & 108 	 & -0.047 & +0.037 & +0.036 & +0.026 \\ 
 1.0e-04 	 & 1.0e-04 	 & 10 & 158 	 & -0.047 & +0.037 & +0.036 & +0.026 \\ 
 1.0e-04 	 & 1.0e-06 	 & 10 & 142 	 & -0.047 & +0.037 & +0.036 & +0.026 \\ 
 \hline
 1.0e-06 	 & 1.0e+02 	 & 10 & 203 	 & +0.636 & +0.108 & +0.035 & +0.032 \\ 
 1.0e-06 	 & 1.0e+00 	 & 10 & 117 	 & -0.035 & +0.039 & +0.035 & +0.026 \\ 
 1.0e-06 	 & 1.0e-02 	 & 10 & 120 	 & -0.047 & +0.037 & +0.036 & +0.026 \\ 
 1.0e-06 	 & 1.0e-04 	 & 10 & 122 	 & -0.047 & +0.037 & +0.036 & +0.026 \\ 
 1.0e-06 	 & 1.0e-06 	 & 10 & 133 	 & -0.047 & +0.037 & +0.036 & +0.026 \\ 

\end{tabular}
\label{Tab::1}
\end{table} 










  





 

 








\begin{table}
\centering
\caption{ Shows the relaxation parameters $\alpha$ and $\beta$, number of timesteps $k$, the resulting number of iterations, the relative error of the estimated diffusion coefficients and the relative error for $g$. The observations times were chosen $ t_i=[2.4, 4.8,$ 7.2, 9.6, 12.0, 14.4, $16.8, 19.2, 21.6, 24.0]$ }
\begin{tabular}{*{8}c}
$\alpha$ & $\beta$ & k & iter & $ D_{\Omega_1}\sp{rel}$& $D_{\Omega_2}\sp{rel} $ & $D_{\Omega_3}\sp{rel} $&$|| g ||\sp{rel} $ \\
\hline
 1.0e+00 	 & 1.0e+02 	 & 20 & 600 	 & +6.412 & +0.175 & +0.086 & +0.124 \\
 1.0e+00 	 & 1.0e+02 	 & 40 &  -   & +6.702 & +0.130 & +0.082 & +0.125 \\  
 1.0e+00 	 & 1.0e+00 	 & 20 & 300 	 & +5.538 & +0.324 & +0.148 & +0.128 \\ 
 1.0e+00 	 & 1.0e+00 	 & 40 & 346 	 & +11.392 & +0.873 & +0.242 & +0.154 \\ 
 1.0e+00 	 & 1.0e-02 	 & 20 & 415 	 & +18.156 & +1.378  & +0.389  & +0.251 \\ 
 1.0e+00 	 & 1.0e-02 	 & 40 & 856 	 & +64.050 & +17.408 & +15.950 & +0.614 \\ 
 1.0e+00 	 & 1.0e-04 	 & 20 & 417 	 & +17.747 & +1.465  & +0.406  & +0.258 \\ 
 1.0e+00 	 & 1.0e-04 	 & 40 & 946 	 & +72.594 & +17.985 & +16.702 & +0.641 \\
 1.0e+00 	 & 1.0e-06 	 & 20 & 399 	 & +17.407 & +1.466  & +0.406  & +0.258 \\ 
 1.0e+00 	 & 1.0e-06 	 & 40 & 863 	 & +60.448 & +18.043 & +16.732 & +0.641 \\
 \hline
 1.0e-02 	 & 1.0e+02 	 & 20 & 351 	 & +0.865 & +0.001 & +0.017 & +0.027 \\ 
 1.0e-02 	 & 1.0e+02 	 & 40 & 404 	 & +0.846 & -0.049 & +0.010 & +0.026 \\ 
 1.0e-02 	 & 1.0e+00 	 & 20 & 254 	 & +0.018 & +0.007 & +0.008 & +0.007 \\ 
 1.0e-02 	 & 1.0e+00 	 & 40 & 218 	 & +0.020 & -0.006 & +0.001 & +0.003 \\ 
 1.0e-02 	 & 1.0e-02 	 & 20 & 381 	 & +0.127 & +0.057 & -0.001 & +0.016 \\ 
 1.0e-02 	 & 1.0e-02 	 & 40 & 543 	 & +0.099 & +0.087 & +0.002 & +0.023 \\
 1.0e-02 	 & 1.0e-04 	 & 20 & 641 	 & +0.171 & +0.082 & -0.003 & +0.091 \\ 
 1.0e-02 	 & 1.0e-04 	 & 40 & 879 	 & +0.303 & +0.383 & +0.079 & +0.222 \\ 
 1.0e-02 	 & 1.0e-06 	 & 20 & 547 	 & +0.173 & +0.084 & -0.003 & +0.092 \\
 1.0e-02 	 & 1.0e-06 	 & 40 & 844 	 & +0.332 & +0.416 & +0.095 & +0.239 \\ 
 \hline 
 1.0e-04 	 & 1.0e+02 	 & 20 & 257 	 & +0.853 & -0.001 & +0.017 & +0.026 \\ 
 1.0e-04 	 & 1.0e+02 	 & 40 & 532 	 & +0.822 & -0.051 & +0.010 & +0.026 \\ 
 1.0e-04 	 & 1.0e+00 	 & 20 & 164 	 & +0.008 & +0.003 & +0.008 & +0.007 \\ 
 1.0e-04 	 & 1.0e+00 	 & 40 & 203 	 & +0.005 & -0.012 & +0.001 & +0.003 \\  
 1.0e-04 	 & 1.0e-02 	 & 20 & 313 	 & +0.071 & +0.021 & -0.001 & +0.007 \\ 
 1.0e-04 	 & 1.0e-02 	 & 40 & 294 	 & +0.004 & +0.004 & -0.001 & +0.001 \\ 
 1.0e-04 	 & 1.0e-04 	 & 20 & 265 	 & +0.108 & +0.035 & -0.002 & +0.014 \\
 1.0e-04 	 & 1.0e-04 	 & 40 & 401 	 & -0.010 & -0.002 & -0.002 & +0.004 \\ 
 1.0e-04 	 & 1.0e-06 	 & 20 & 452 	 & +0.066 & +0.021 & -0.003 & +0.031 \\ 
 1.0e-04 	 & 1.0e-06 	 & 40 & 330 	 & -0.014 & -0.013 & -0.003 & +0.004 \\ 
 \hline
 1.0e-06 	 & 1.0e+02 	 & 20 & 274 	 & +0.850 & -0.001 & +0.017 & +0.026 \\ 
 1.0e-06 	 & 1.0e+02 	 & 40 & 496 	 & +0.821 & -0.051 & +0.010 & +0.026 \\
 1.0e-06 	 & 1.0e+00 	 & 20 & 176 	 & +0.008 & +0.003 & +0.008 & +0.007 \\ 
 1.0e-06 	 & 1.0e+00 	 & 40 & 207 	 & +0.006 & -0.012 & +0.001 & +0.003 \\ 
 1.0e-06 	 & 1.0e-02 	 & 20 & 223 	 & +0.075 & +0.024 & -0.001 & +0.007 \\ 
 1.0e-06 	 & 1.0e-02 	 & 40 & 392 	 & +0.000 & -0.001 & -0.001 & +0.002 \\ 
 1.0e-06 	 & 1.0e-04 	 & 20 & 429 	 & +0.085 & +0.027 & -0.002 & +0.025 \\ 
 1.0e-06 	 & 1.0e-04 	 & 40 & 241 	 & -0.020 & -0.030 & -0.005 & +0.005 \\ 
 1.0e-06 	 & 1.0e-06 	 & 20 & 591 	 & +0.060 & +0.021 & -0.003 & +0.048 \\ 
 1.0e-06 	 & 1.0e-06 	 & 40 & 343 	 & -0.014 & -0.010 & -0.003 & +0.004 \\ 

\end{tabular}
\label{TAB::timesteps}
\end{table} 


\begin{table}
\centering
\caption{ Shows the relaxation parameters $\alpha$ and $\beta$, number of timesteps $k$, the resulting number of iterations, the relative error of the estimated diffusion coefficients and the relative error for $g$. The observation times were set $t_i = [ 4.8, 9.6, 14.4, 19.2, 24.0 ] $. }
\begin{tabular}{*{8}c}
$\alpha$ & $\beta$ & k & iter & $ D_{\Omega_1}\sp{rel}$& $D_{\Omega_2}\sp{rel} $ & $D_{\Omega_3}\sp{rel} $&$|| g ||\sp{rel} $ \\
\hline
 1.0e-02 	 & 1.0e+00 	 & 10 & 151 	 & +0.095 & +0.015 & +0.028 & +0.020 \\ 
 1.0e-02 	 & 1.0e+00 	 & 20 & 165 	 & +0.092 & -0.001 & -0.001 & +0.008 \\ 
 1.0e-02 	 & 1.0e+00 	 & 40 & 228 	 & +0.067 & -0.024 & -0.011 & +0.005 \\ 
 
 1.0e-02 	 & 1.0e-02 	 & 10 & 330 	 & +0.190 & +0.074 & +0.027 & +0.083 \\ 
 1.0e-02 	 & 1.0e-02 	 & 20 & 455 	 & +3.882 & +3.297 & +2.583 & +0.304 \\ 
  1.0e-02 	 & 1.0e-02 	 & 40 & 523 	 & +7.466 & +6.454 & +5.537 & +0.388 \\ 

 1.0e-02 	 & 1.0e-04 	 & 10 & 428 	 & +0.196 & +0.076 & +0.032 & +0.163 \\ 
 1.0e-02 	 & 1.0e-04 	 & 20 & 571 	 & +7.402 & +6.421 & +5.405 & +0.680 \\ 
 1.0e-02 	 & 1.0e-04 	 & 40 & 747 	 & +15.386 & +13.764 & +12.196 & +0.841 \\ 
  
  1.0e-04 	 & 1.0e+00 	 & 10 & 147 	 & +0.081 & +0.008 & +0.023 & +0.020 \\ 
  1.0e-04 	 & 1.0e+00 	 & 20 & 167 	 & +0.051 & -0.026 & -0.009 & +0.008 \\ 
  1.0e-04 	 & 1.0e+00 	 & 40 & 148 	 & +0.018 & -0.057 & -0.020 & +0.005 \\ 
  
 1.0e-04 	 & 1.0e-02 	 & 10 & 255 	 & +0.171 & +0.087 & +0.010 & +0.021 \\ 
 1.0e-04 	 & 1.0e-02 	 & 20 & 216 	 & -0.005 & -0.006 & -0.008 & +0.008 \\ 
 1.0e-04 	 & 1.0e-02 	 & 40 & 386 	 & -0.004 & -0.012 & -0.015 & +0.004 \\ 
  
  
 1.0e-04 	 & 1.0e-04 	 & 10 & 284 	 & +0.059 & +0.052 & +0.010 & +0.051 \\ 
 1.0e-04 	 & 1.0e-04 	 & 20 & 256 	 & -0.048 & -0.045 & -0.024 & +0.020 \\ 
  1.0e-04 	 & 1.0e-04 	 & 40 & 277 	 & -0.071 & -0.074 & -0.050 & +0.018 \\ 
  
 1.0e-06 	 & 1.0e+00 	 & 10 & 163 	 & +0.082 & +0.008 & +0.023 & +0.020 \\ 
 1.0e-06 	 & 1.0e+00 	 & 20 & 156 	 & +0.054 & -0.027 & -0.009 & +0.008 \\ 
 1.0e-06 	 & 1.0e+00 	 & 40 & 284 	 & +0.020 & -0.055 & -0.019 & +0.005 \\
  
 1.0e-06 	 & 1.0e-02 	 & 10 & 290 	 & +0.155 & +0.075 & +0.011 & +0.021 \\ 
 1.0e-06 	 & 1.0e-02 	 & 20 & 239 	 & -0.004 & +0.002 & -0.006 & +0.008 \\ 
 1.0e-06 	 & 1.0e-02 	 & 40 & 378 	 & -0.005 & -0.013 & -0.016 & +0.004 \\ 
  
 1.0e-06 	 & 1.0e-04 	 & 10 & 264 	 & +0.070 & +0.056 & +0.009 & +0.050 \\ 
 1.0e-06 	 & 1.0e-04 	 & 20 & 234 	 & -0.047 & -0.044 & -0.029 & +0.020 \\
 1.0e-06 	 & 1.0e-04 	 & 40 & 324 	 & -0.070 & -0.070 & -0.052 & +0.017 \\ 

\end{tabular}
\label{TAB::half}
\end{table} 

\begin{table}
\centering
\caption{ Shows the relaxation parameters $\alpha$ and $\beta$, number of timesteps $k$, the resulting number of iterations, the relative error of the estimated diffusion coefficients and the relative error for $g$. The observation times were $t_i = [ $1.2, 2.4, 3.6, 4.8, 6.0, 7.2, 8.4, 9.6, 10.8, 12.0, 13.2, 14.4, 15.6, 16.8, 17.0, 19.2, 20.4,$ 21.6, 22.8, 24.0 ]$.}
\begin{tabular}{*{8}c}
$\alpha$ & $\beta$ & k & iter & $ D_{\Omega_1}\sp{rel}$ & $ D_{\Omega_2}\sp{rel}$ & $D_{\Omega_3}\sp{rel} $ & $|| g ||\sp{rel} $ \\
\hline

 1.0e-02 	 & 1.0e+00 	 & 10 & 250 	 & -0.038 & +0.035 & +0.019 & +0.033 \\ 
 1.0e-02 	 & 1.0e+00 	 & 20 & 225 	 & -0.002 & +0.015 & +0.007 & +0.007 \\ 
 1.0e-02 	 & 1.0e+00 	 & 40 & 255 	 & +0.003 & +0.002 & +0.001 & +0.002 \\ 

 1.0e-02 	 & 1.0e-02 	 & 10 & 180 	 & -0.042 & +0.034 & +0.020 & +0.033 \\ 
 1.0e-02 	 & 1.0e-02 	 & 20 & 445 	 & +0.010 & +0.017 & +0.003 & +0.010 \\ 
  1.0e-02 	 & 1.0e-02 	 & 40 & 429 	 & +0.037 & +0.014 & -0.003 & +0.004 \\ 
  
 1.0e-02 	 & 1.0e-04 	 & 10 & 208 	 & -0.042 & +0.034 & +0.020 & +0.033 \\ 
  1.0e-02 	 & 1.0e-04 	 & 20 & 530 	 & +0.012 & +0.017 & +0.003 & +0.020 \\ 
 1.0e-02 	 & 1.0e-04 	 & 40 & - 	 & +0.064 & +0.023 & -0.005 & +0.072 \\ 
 
 1.0e-04 	 & 1.0e+00 	 & 10 & 209 	 & -0.043 & +0.033 & +0.019 & +0.033 \\ 
 1.0e-04 	 & 1.0e+00 	 & 20 & 212 	 & -0.007 & +0.014 & +0.007 & +0.007 \\ 
 1.0e-04 	 & 1.0e+00 	 & 40 & 258 	 & -0.002 & +0.001 & +0.001 & +0.002 \\ 
 
 1.0e-04 	 & 1.0e-02 	 & 10 & 258 	 & -0.047 & +0.033 & +0.020 & +0.033 \\ 
 1.0e-04 	 & 1.0e-02 	 & 20 & 391 	 & +0.005 & +0.016 & +0.003 & +0.008 \\ 
 1.0e-04 	 & 1.0e-02 	 & 40 & 261 	 & +0.014 & +0.005 & -0.002 & +0.001 \\ 
 
 1.0e-04 	 & 1.0e-04 	 & 10 & 267 	 & -0.047 & +0.033 & +0.020 & +0.033 \\ 
 1.0e-04 	 & 1.0e-04 	 & 20 & 769 	 & +0.008 & +0.016 & +0.003 & +0.016 \\
 1.0e-04 	 & 1.0e-04 	 & 40 & 375 	 & +0.027 & +0.005 & -0.003 & +0.002 \\ 
 
 1.0e-06 	 & 1.0e+00 	 & 10 & 198 	 & -0.042 & +0.033 & +0.019 & +0.033 \\ 
 1.0e-06 	 & 1.0e+00 	 & 20 & 222 	 & -0.007 & +0.014 & +0.007 & +0.007 \\ 
 1.0e-06 	 & 1.0e+00 	 & 40 & 280 	 & -0.001 & +0.001 & +0.001 & +0.002 \\ 
 
 1.0e-06 	 & 1.0e-02 	 & 10 & 279 	 & -0.047 & +0.033 & +0.020 & +0.033 \\ 
 1.0e-06 	 & 1.0e-02 	 & 20 & 312 	 & +0.005 & +0.016 & +0.003 & +0.008 \\ 
 1.0e-06 	 & 1.0e-02 	 & 40 & 379 	 & +0.015 & +0.005 & -0.002 & +0.001 \\
 
 1.0e-06 	 & 1.0e-04 	 & 10 & 251 	 & -0.047 & +0.033 & +0.020 & +0.033 \\ 
 1.0e-06 	 & 1.0e-04 	 & 20 & 778 	 & +0.008 & +0.016 & +0.002 & +0.018 \\ 
 1.0e-06 	 & 1.0e-04 	 & 40 & 191 	 & +0.017 & +0.004 & -0.002 & +0.002 \\ 
 

\end{tabular}
\label{TAB::double}
\end{table} 


\newpage
\begin{table}
\centering
\caption{Shows the relaxation parameters $\alpha$ and $\beta$, number of timesteps $k$, the resulting number of iterations, the relative error of the estimated optimal parameters for the diffusion coefficients and the relative error for $g$. The noise amplitude was set to 0.03, and $ t_i=[2.4, 4.8, 7.2, 9.6, 12.0, 14.4, 16.8, 19.2, 21.6, 24.0]$ }
\begin{tabular}{*{8}c}
$\alpha$ & $\beta$ & k  & iter & $ D_{\Omega_1}\sp{rel} $ & $D_{\Omega_2}\sp{rel}$ & $D_{\Omega_3}\sp{rel}$ & $|| g ||\sp{rel} $\\
\hline
 1.0e-02 	 & 1.0e+00 	 & 10 & 151 	 & -0.030 & +0.041 & +0.036 & +0.026 \\ 
 1.0e-02 	 & 1.0e-02 	 & 10 & 120 	 & -0.032 & +0.039 & +0.036 & +0.026 \\ 
 1.0e-02 	 & 1.0e-04 	 & 10 & 125 	 & -0.038 & +0.039 & +0.037 & +0.026 \\ 
 1.0e-02 	 & 1.0e-06 	 & 10 & 167 	 & -0.040 & +0.039 & +0.036 & +0.026 \\ 
 1.0e-04 	 & 1.0e+00 	 & 10 & 161 	 & -0.038 & +0.039 & +0.035 & +0.026 \\ 
 1.0e-04 	 & 1.0e-02 	 & 10 & 95 	 & -0.047 & +0.036 & +0.036 & +0.026 \\ 
 1.0e-04 	 & 1.0e-04 	 & 10 & 75 	 & -0.041 & +0.036 & +0.035 & +0.026 \\ 
 1.0e-04 	 & 1.0e-06 	 & 10 & 136 	 & -0.044 & +0.036 & +0.036 & +0.026 \\ 
 1.0e-06 	 & 1.0e+00 	 & 10 & 134 	 & -0.034 & +0.037 & +0.037 & +0.026 \\ 
 1.0e-06 	 & 1.0e-02 	 & 10 & 112 	 & -0.046 & +0.038 & +0.036 & +0.026 \\ 
 1.0e-06 	 & 1.0e-04 	 & 10 & 111 	 & -0.043 & +0.036 & +0.037 & +0.026 \\ 
 1.0e-06 	 & 1.0e-06 	 & 10 & 214 	 & -0.049 & +0.037 & +0.036 & +0.026 \\
  
 1.0e-02 	 & 1.0e+00 	 & 20 & 190 	 & +0.013 & +0.006 & +0.008 & +0.007 \\ 
 1.0e-02 	 & 1.0e-02 	 & 20 & 392 	 & +0.126 & +0.057 & +0.001 & +0.017 \\ 
 1.0e-02 	 & 1.0e-04 	 & 20 & 677 	 & +0.164 & +0.074 & -0.001 & +0.093 \\ 
 1.0e-02 	 & 1.0e-06 	 & 20 & 600 	 & +0.179 & +0.081 & -0.001 & +0.098 \\ 
 1.0e-04 	 & 1.0e+00 	 & 20 & 224 	 & +0.006 & +0.004 & +0.008 & +0.007 \\ 
 1.0e-04 	 & 1.0e-02 	 & 20 & 465 	 & +0.074 & +0.022 & -0.003 & +0.008 \\ 
 1.0e-04 	 & 1.0e-04 	 & 20 & 586 	 & +0.071 & +0.023 & -0.005 & +0.056 \\ 
 1.0e-04 	 & 1.0e-06 	 & 20 & 472 	 & +0.081 & +0.024 & -0.003 & +0.056 \\ 
 1.0e-06 	 & 1.0e+00 	 & 20 & 157 	 & +0.011 & +0.003 & +0.009 & +0.007 \\ 
 1.0e-06 	 & 1.0e-02 	 & 20 & 352 	 & +0.076 & +0.021 & +0.000 & +0.008 \\ 
 1.0e-06 	 & 1.0e-04 	 & 20 & 738 	 & +0.083 & +0.026 & -0.004 & +0.065 \\ 
 1.0e-06 	 & 1.0e-06 	 & 20 & 599 	 & +0.056 & +0.021 & -0.003 & +0.106 \\ 
 

\end{tabular}
\label{Tab::Noise0.03}
\end{table} 
\begin{table}[t]
\centering
\caption{Shows the relaxation parameters $\alpha$ and $\beta$, number of timesteps $k$, the resulting number of iterations, the relative error of the estimated optimal parameters for the diffusion coefficients and the relative error for $g$. The noise amplitude was set to 0.3, and $t_i =[2.4, 4.8, 7.2, 9.6, 12.0, 14.4, 16.8, 19.2, 21.6, 24.0]$  }
\begin{tabular}{*{8}c}
$\alpha$ & $\beta$ & k  & iter & $ D_{\Omega_1}\sp{rel}$ & $D_{\Omega_2}\sp{rel} $ & $D_{\Omega_3}\sp{rel} $ & $|| g ||\sp{rel} $ \\
\hline
 1.0e-02 	 & 1.0e+00 	 & 10 & 144 	 & +0.012 & +0.055 & +0.027 & +0.036 \\ 
 1.0e-02 	 & 1.0e-02 	 & 10 & 476 	 & -0.009 & +0.049 & +0.035 & +0.051 \\ 
 1.0e-02 	 & 1.0e-04 	 & 10 & 476 	 & -0.097 & +0.042 & +0.054 & +0.054 \\ 
 1.0e-02 	 & 1.0e-06 	 & 10 & 558 	 & -0.071 & +0.047 & +0.039 & +0.054 \\ 
 1.0e-04 	 & 1.0e+00 	 & 10 & 182 	 & -0.040 & +0.033 & +0.051 & +0.037 \\ 
 1.0e-04 	 & 1.0e-02 	 & 10 & 471 	 & +0.021 & +0.038 & +0.036 & +0.056 \\ 
 1.0e-04 	 & 1.0e-04 	 & 10 & 792 	 & -0.058 & +0.042 & +0.031 & +0.565 \\ 
 1.0e-04 	 & 1.0e-06 	 & 10 &  -   & -0.050 & +0.029 & +0.040 & +1.062 \\ 
 1.0e-06 	 & 1.0e+00 	 & 10 & 178 	 & +0.012 & +0.033 & +0.021 & +0.036 \\ 
 1.0e-06 	 & 1.0e-02 	 & 10 & 537 	 & -0.018 & +0.040 & +0.025 & +0.057 \\ 
 1.0e-06 	 & 1.0e-04 	 & 10 &  -	 & -0.038 & +0.040 & +0.030 & +1.452 \\ 
 1.0e-06 	 & 1.0e-06 	 & 10 &  - 	 & +0.003 & +0.035 & +0.046 & +3.621 \\ 
 1.0e-02 	 & 1.0e+00 	 & 20 & 213 	 & +0.009 & +0.015 & +0.009 & +0.016 \\ 
 1.0e-02 	 & 1.0e-02 	 & 20 & 719 	 & +0.228 & +0.088 & -0.032 & +0.085 \\ 
 1.0e-02 	 & 1.0e-04 	 & 20 &  -   & +0.100 & +0.071 & -0.013 & +0.271 \\ 
 1.0e-02 	 & 1.0e-06 	 & 20 &   -  & +0.223 & +0.097 & -0.027 & +0.282 \\ 
 1.0e-04 	 & 1.0e+00 	 & 20 & 245 	 & +0.038 & +0.013 & -0.001 & +0.016 \\ 
 1.0e-04 	 & 1.0e-02 	 & 20 &  - 	 & +0.050 & +0.054 & -0.024 & +0.116 \\ 
 1.0e-04 	 & 1.0e-04 	 & 20 &  - 	 & -0.087 & +0.059 & -0.077 & +5.763 \\ 
 1.0e-04 	 & 1.0e-06 	 & 20 &  -	 & +0.001 & +0.083 & -0.038 & +7.485 \\ 
 1.0e-06 	 & 1.0e+00 	 & 20 & 242 	 & -0.009 & +0.008 & +0.015 & +0.016 \\ 
 1.0e-06 	 & 1.0e-02 	 & 20 & 947 	 & +0.122 & +0.093 & -0.040 & +0.115 \\ 
 1.0e-06 	 & 1.0e-04 	 & 20 &  - 	 & -0.076 & +0.041 & -0.020 & +7.700 \\ 
 1.0e-06 	 & 1.0e-06 	 & 20 &  - 	 & +0.147 & +0.113 & -0.023 & +10.936 \\ 

\end{tabular}
\label{Tab::Noise0.3}
\end{table} 


\begin{table}
\centering
\caption{Shows the relaxation parameters $\alpha$ and $\beta$, number of timesteps $k$, the resulting number of iterations, the estimated diffusion coefficients for grey $D_{\Omega_{GM}}$ and white $D_{\Omega_{WM}}$ matter based on MRI data.}
\begin{tabular}{*{6}c}
$\alpha$ & $\beta$ & k & iter &  $ D_{\Omega_{GM}} \left[ \mathrm{mm\sp{2}/h} \right] $ & $ D_{\Omega_{WM}} \left[ \mathrm{mm\sp{2}/h} \right]$ \\
\hline
 1.0e-02 	 & 1.0e+00 	 & 24 & 503 	 & 0.659 & 0.918 \\ 
 1.0e-02 	 & 1.0e-01 	 & 24 & 716 	 & 0.771 & 0.969 \\ 
 1.0e-02 	 & 1.0e-02 	 & 24 & 505 	 & 0.836 & 0.974 \\ 
 1.0e-04 	 & 1.0e+00 	 & 24 & 206 	 & 0.553 & 0.828 \\ 
 1.0e-04 	 & 1.0e-01 	 & 24 & 440 	 & 0.651 & 0.789 \\ 
 1.0e-04 	 & 1.0e-02 	 & 24 & 597 	 & 0.718 & 0.739 \\ 
 1.0e-06 	 & 1.0e+00 	 & 24 & 439 	 & 0.570 & 0.830 \\ 
 1.0e-06 	 & 1.0e-01 	 & 24 & 583 	 & 0.656 & 0.793 \\ 
 1.0e-06 	 & 1.0e-02 	 & 24 & 742 	 & 0.721 & 0.729 \\ 
 
 %1.0e-02    & 1.0e+02   & 48 & 714      & 0.564 & 0.952 ?? \\ 
 1.0e-02 	 & 1.0e+00 	 & 48 & 591 	 & 0.648 & 0.893 \\ 
 1.0e-02 	 & 1.0e-01 	 & 48 & 461 	 & 0.772 & 0.982 \\ 
 1.0e-02 	 & 1.0e-02 	 & 48 & 748 	 & 0.986 & 1.198 \\ 
 %1.0e-04    & 1.0e+02   & 48 & 815      & 0.470 & 0.928 \\ 
 1.0e-04 	 & 1.0e+00 	 & 48 & 607 	 & 0.551 & 0.815 \\ 
 1.0e-04 	 & 1.0e-01 	 & 48 & 598 	 & 0.649 & 0.787 \\ 
 1.0e-04 	 & 1.0e-02 	 & 48 & 837 	 & 0.740 & 0.817 \\
 %1.0e-06    & 1.0e+02   & 48 & 729      & 0.428 & 0.948 \\ 
 1.0e-06 	 & 1.0e+00 	 & 48 & 681 	 & 0.557 & 0.811 \\ 
 1.0e-06 	 & 1.0e-01 	 & 48 & 780 	 & 0.645 & 0.774 \\ 
 1.0e-06 	 & 1.0e-02 	 & 48 & 745 	 & 0.736 & 0.819 \\ 



\end{tabular}
\label{Tab::Real-data}
\end{table} 
 

\end{document}


