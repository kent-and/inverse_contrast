\clearpage
\section{Supplementary}

\subsection{Verification in terms of the manufactured solution}
Below we will discuss the parameter identification and its sensitivity with respect to the regularization parameters, noise, number of observations and time-resolution of the forward model. When presenting the results, we denote the relative error of the diffusion parameters as 
\begin{equation}
 D_{\Omega_i}\sp{rel} = \frac{D_{\Omega_i}\sp{found} -D_{\Omega_i}\sp{true} }{D_{\Omega_i}\sp{true} }, \text{ for } i=1,2,3
\end{equation}
For the boundary parameter $g$, the relative error norm is defined as 
\begin{equation}
|| g ||\sp{rel} = \frac{ \sum\limits_{t_j} || g_{found}(t_j) -g_{true}(t_j)||_{L^2(\Omega_1)} }{  \sum\limits_{t_j}||g_{true}(t_j)||_{L^2(\Omega_1)} }
\end{equation}
The case when the minimization algorithm fails to converge within 1000 iteration will be indicated by a hyphen in Table~\ref{Tab::Noise0.3}.



\subsubsection{The relaxation parameters}
The convergence as well as the result of the minimization is influenced by the values of the regularization parameters $\alpha$ and $\beta$. Therefore the convergence was examined by a systematic study of the reconstruction of the manufactured solution with respect to a wide range of regularization parameters. 

Ideally, our approach should have a wide range of parameters in which the reconstruction algorithm yields very similar end-results although the convergence may vary substantially. The evaluation of different regularization parameters was done by solving the optimization with different values of $\alpha$ and $\beta$ and comparing the computed $D$ and $g$ values with the manufactured solution. The observation times were set to $\lbrace$2.4, 4.8, 7.2, 9.6, 12.0, 14.4, 16.8, 19.2, 21.6, 24.0 $\rbrace$, and the number of time-steps $k$ in the forward problem was set to $10$. 

The results are shown in Table~\ref{Tab::1} and the convergence is shown in Figure~\ref{convergence}. We can see that the optimization computed similar end-result for most regularization parameters. More specifically, for $\alpha \leq 1\mathrm{e}{-2}$ and $\beta \leq 1e-0$ the relative errors for all optimization parameters are below $5\%$.However, in the case of $ \alpha=1.0$ the relative error is over 100\% higher than for other values of $\alpha$. Furthermore, for all $\beta=100.0$ there is about 70\% relative error for $D_{\Omega_1}$ and approximate 10\% error for $D_{\Omega_2}$. From the results it can be discerned that $\alpha =1.0$ and $\beta=100.0$ represent an upper threshold where the observations are no longer the dominant term in the objective functional.  


\begin{figure}
\centering
\includegraphics[scale=0.18]{../Convergence-plot.png}   
\caption{Convergence plots of the diffusion coefficients, boundary conditions and functional with respect to different $\alpha$ and $\beta$ values. } 
\label{convergence}
\end{figure}

In Figure~\ref{boundarycontrol} we can see that more time-steps causes oscillations if $\alpha\gg\beta$ and the number of time-steps are much larger than the number of observations. For instance, as seen by the the pink line for $\alpha=1.0$ and $\beta=0.0001$ the peaks corresponds to a observation. These oscillations are caused by the strong weighting of $\alpha$, which leads to minimizing of $g$ between observation times. 
Tests with variation in the number of timesteps and observations are documented in the Supplementary. 

%It can be seen that more time-steps causes an increase number of iterations, but for some values the number of iterations decreases. These values share that $\alpha $ is a magnitude 2 lower than $ \beta$,i.e $\alpha=1.0e-2$ and $\beta =1.0$. This indicates that $\alpha $ should be smaller than $\beta$ to achieve optimal convergence. However, the number of iterations in Table ~\ref{Tab::1} are smaller for all regularization parameters. 

\begin{figure}
\centering
\includegraphics[scale=0.21]{../boundary_control.png}  
\caption{ Displays plots over time for a selection of points at the boundary of $\Omega_1$ with different regularization parameters and number of timesteps $k$. The left panel shows the legend for the plot over time, together with the selection of points. The middle panel shows the average boundary value $g$ for different regularization parameter with $k=10$. The right panel shows the average boundary value $g$ for different regularization parameter with $k=40$.  }
\label{boundarycontrol}
\end{figure}

\subsubsection{The noise susceptibility}
The MRI data contains noise, hence an investigation to the noise susceptibility is needed. This was done varying the noise amplitude, and then the optimization was performed for various regularization parameters $\alpha$ and $\beta$. Figure~\ref{12hourswithnoise} and Figure~\ref{24hourswithnoise} show the reconstruction of the manufactured solution at
different noise levels with $\alpha=0.0001$, $\beta=1.0$ and $k=20$. Clearly, the reconstruction shown in the lower rows is quite robust with respect to the various levels
of noise shown in the upper rows.      
In Table~\ref{Tab::Noise0.3} the noise amplitude was set to 0.3 (100$\%$ of maximum initial condition), which gave a visible effect. The noise caused the number of iteration to increase, and with more timesteps caused the optimization not to convergence for $\beta < 1.0$. Furthermore, the relative boundary error $||g||\sp{rel}$ is significantly larger for $\alpha< 1.0\mathrm{e}{-2} $ and $\beta < 1.0\mathrm{e}{-2}$. Results with noise  amplitudes at 0.03 are listed in the Supplementary. 


\begin{figure}
\centering
\includegraphics[scale=0.4]{../noise-12.png}  
\caption{The upper row shows the manufactured observation, A) Shows the manufactured observation at time-point 24 with no noise added. B) Shows the manufactured observation at time-point 24 with noise amplitude of 0.15. C) Shows the manufactures observation at time-point 24 with noise amplitude of 0.3. The lower row shows the results with optimized parameter obtained with $\alpha=0.0001$, $\beta=1.0$ and $k=20$. D) Shows the resulting state given the observation in A. E)  Shows the resulting state given the observation in B .F) Shows the resulting state given the observation in C. }
\label{12hourswithnoise}
\end{figure}


\begin{figure}
\centering
\includegraphics[scale=0.4]{../noise-24.png}
\caption{The upper row shows the manufactured observation, A) Shows the manufactured observation at time-point 24 with no noise added. B) Shows the manufactured observation at time-point 24 with noise amplitude of 0.15. C) Shows the manufactures observation at time-point 24 with noise amplitude of 0.3. The lower row shows the results with optimized parameter obtained with $\alpha=0.0001$, $\beta=1.0$ and $k=20$. D) Shows the resulting state given the observation in A. E)  Shows the resulting state given the observation in B .F) Shows the resulting state given the observation in C.  }
\label{24hourswithnoise}
\end{figure}



\subsection{Manufactured Solution}
In order to assess the robustness and accuracy of our strategy and test the dependency of the computed diffusion coefficients $D$ on the
regularization parameters $\alpha$ and $\beta$, we perform a test case with a
known solution. The manufactured observations were obtained by forward computation of \eqref{Eq::PDE} with the Dirichlet boundary condition defined as 
\begin{equation}
g(t) = 0.3 +0.167t - 0.007t\sp{2} \qquad \text{ for } 0 \leq t \leq 24.
\label{EQ::DIRI}
\end{equation}
The initial condition was set to 0 everywhere, the timestep was $dt = 0.2$, and the diffusion coefficients were selected to be 
\begin{equation}
D_{\Omega_1} = 1000.0, \quad D_{\Omega_2} = 4.0, \quad D_{\Omega_3} = 8.0 
\end{equation}  
The magnitude order for $D_{\Omega_1} $ and $D_{\Omega_2}$ were chosen to resemble diffusion coefficient for water in grey and white matter, but the relation between the coefficients were not preserved. In \citet{Haga}, the drug concentration moved 1 cm per 20 seconds in the cervical region, thus $D_{\Omega_1}$  was set to 1000. The manufactured forward computation produced a total of 120 possible observation times.


\subsection{Implementation}
The solver for \eqref{Eq::PDE} was implemented using the FEniCS project (v.2017.2), using backwards Euler time discretization and first order continuous Galerkin finite elements in space. The resulting linear problems were solved using GMRES.
The module dolfin-adjoint \cite{farrell2013automated, funke2013framework} was used to automatically derive and solve the adjoint problem, and to solve the PDE constrained optimization problem with the L-BFGS-B algorithm \cite{LBFGSB1, LBFGSB2}. The optimization was stopped when the $L^\infty$-norm of the projected gradient of the objective functional dropped below $1.0\mathrm{e}{-1}$.

The observations $u_{obs}$ have fixed time-points. However, these time-points may not coincide  with the time discretization in the forward problem $t_j$. Therefore, we linearly interpolated the computed solutions onto the observation time points using the formula
\begin{equation}
\label{observation:interpolation}
u(t) \approx \frac{\Delta t}{dt} u(t_{j-1}) + \frac{dt - \Delta t }{dt} u(t_{j}), \quad t \in \lbrace t_{j-1}, t_j \rbrace
\end{equation}
with $\Delta t = t_j-t$ and $dt = t_j - t_{j-1}$.

The implementation used the first observation as initial conditions of \eqref{Eq::PDE}. Then for each time-step, the next observation was used as the initial value for boundary control $g$. In order to increase the convergence speed of the optimization, $D_{\Omega_1}$ was scaled so that $D_{\Omega_1} = 100 D_{\Omega_1}\sp{\star} $. The initial values of the diffusion coefficients in the optimization algorithm were $(D_{\Omega_1}\sp{\star}, D_{\Omega_2}, D_{\Omega_3})=$  $(1, 1, 1)$. 

The noise susceptibility was tested by pointwise adding a uniform distributed noise to the observations. The noise term was constructed using the random library in numpy and adjusted so that the noise range was $\lbrace -n_{amp} , n_{amp} \rbrace $, with $n_{amp}$ denoted as noise amplitude.




\subsection{Contrast concentration - Image Signal Relation}
Below, we briefly describe the relationship between the imaging signal 
seen in Figures~\ref{fig1} and \ref{fig2} and the underlying contrast 
concentration. We remark that we use a notation common in medical literature and here two letter symbols are common. Hence, below we will use two letter symbols such as $TE$ and $TR$ to keep the notation consistent with the presentation in \cite{GOWLAND, MPRAGE}.   
The contrast concentration $c$ causes the longitudinal(spin-lattice) relaxation time $T_{1}$ to shorten with the following relation
\begin{equation}
\frac{1}{T_{1}\sp{c}} = \frac{1}{T_{1}\sp{0}} + r_{1}c .
\label{EQ::contrast}
\end{equation}
The superscripts indicate relaxation time with contrast $T_{1}\sp{c}$ and without contrast $T_{1}\sp{0}$, and $r_1$ is the relaxivity constant for the MRI-contrast in a medium. 
The contrast observations were collected using a MRI sequence known as  Magnetization Prepared Rapid Acquisition Gradient Echo (MPRAGE) with an inversion prepared magnetization. The relation between signal and the relaxation time is non-linear, and is expressed with the following equations. The signal value $S$ for this sequence is given by
\begin{equation}
S = M_{n} \sin \theta e\sp{ - TE/T_2\sp{*} },
\label{EQ::SI_T2}
\end{equation}
with $TE$ and $\theta$ respectively denoting the echo time and the flip angle, and $M_{n}$ the magnetization for the n-echo described below. 
Also $T_2\sp{*}$ is transverse magnetization caused by a combination of spin-spin relaxation and magnetic field inhomogeneity. It is defined as 
\begin{equation}
\frac{1}{T_2\sp{*}} = \frac{1}{T_2} + \gamma \Delta B_{in} ,
\end{equation}
with $T_2$ transverse (spin-spin) relaxation time, $\gamma$ is the gyromagnetic ratio and $\Delta B_{in}$ is the magnetic field inhomogeneity across a voxel. The expression can be simplified by neglecting the $T_2$ term in the signal, since $TE <<T_2\sp{*}$ for this MRI sequence. Thus \eqref{EQ::SI_T2} becomes 
\begin{equation}
S = M_{n} \sin \theta.
\label{EQ::SI}
\end{equation}
In article \cite{GOWLAND}, the term $M_n$ is defined as the magnetization for the n-echo 
\begin{equation}
M_{n} = M_{0}  \left[ (1-\beta)\frac{(1-(\alpha \beta)\sp{n-1} }{1-\alpha\beta} + (\alpha \beta)\sp{n-1}(1-\gamma) + \gamma ( \alpha \beta)\sp{n-1} \frac{M_{e}}{M_{0}}  \right]   
\end{equation}
with 
\begin{equation}
\frac{M_{e}}{M_{0}} = - \left[ \frac{ 1 -\delta + \alpha \delta (1-\beta ) \frac{1-\alpha\beta\sp{m}}{1-\alpha \beta} + \alpha\delta(\alpha\beta)\sp{m-1} - \alpha\sp{m}\rho}{1 +\rho \alpha\sp{m} } \right].
\end{equation}
Using the following definitions
\begin{equation}
\begin{aligned}
\alpha &= \cos ( \theta ) \\
\beta  &= e\sp{- \sp{T_b}/_{T_1\sp{c}} } \\
\delta &= e\sp{- \sp{T_a}/_{T_1\sp{c}} } \\
\gamma &= e\sp{- \sp{T_w}/_{T_1\sp{c}} } \\
\rho   &= e\sp{- \sp{TR}/_{T_1\sp{c}}}  \\
T_w    &= TR - T_a -T_b(m-1)       .\\
\end{aligned}
\end{equation}
Here $T_b$ is known as the echo spacing time, $T_a$ is the inversion time, $T_w$ the time delay, $TR$ as the repetition time, $m$ is the number of echo spacings and $M_0$ is a calibration constant for the magnetization. The center echo denoted as $n=m/2$ will be the signal that we will consider when estimating MRI-contrast. Given \eqref{EQ::SI} we have that the relative signal increase can be written as 
\begin{equation}
\frac{S\sp{c}}{S\sp{0}} = \frac{ M_{n}\sp{c} \sin (\theta)}{ M_{n}\sp{0} \sin (\theta) }.
\end{equation}
We define that  
\begin{equation}
f(T_1) = M_{n}/M_{0} ,
\label{scaledmagnetization}
\end{equation}
which can be seen in Figure~\ref{figuredti}. 
This gives the following relation 
\begin{equation}
\frac{f(T_{1}\sp{c} ) }{f(T_{1}\sp{0})}  = \frac{S\sp{c}}{S\sp{0}} 
\end{equation}
The signal difference between observation times were adjusted in \cite{ringstad2018brain}. Thus we can express the change in $T_1$ due to contrast as 
\begin{equation}
f ( T_{1}\sp{c} ) = \frac{S\sp{c}}{S\sp{0}} f(T_{1}\sp{0}) 
\end{equation}
and then estimate the concentration using \eqref{EQ::contrast}. The $T_{1}\sp{0}$ values were obtained by T1-mapping of the brain using a MRI sequence known as MOLLI5(3)3 \cite{TAYLOR201667}. This takes into account patient specific characteristic, such as tissue damage. Tissue damage can be observed in the MRI due to a lower signal in the white matter compared to healthy white matter tissue, thus damaged tissue have different $T_1$ relaxation time. 
The contrast concentration was estimated in a preprocessing step, using the parameters obtained from the T1-map, MPRAGE MRI protocol \cite{ringstad2018brain} and the value for $r_1$ found in \cite{pmid16230904}. In the computation, the function \eqref{scaledmagnetization} was computed for $ T_1\in \lbrace 200, 4000 \rbrace$ creating a lookup table. The lookup table was utilized with the baseline signal increase to estimate $T_1\sp{c}$, and then the concentration was computed using \eqref{EQ::contrast}.  

\begin{figure}
\centering
\includegraphics[width=0.70\textwidth]{../T1function.png} 
\caption{Shows the  function defined in \eqref{scaledmagnetization} where the white region indicates $T_1$ values for white matter, the grey region indicates $T_1$ values for grey matter, the blue region indicates $T_1$ values for CSF.  }
\label{figureF} 
\end{figure}


\subsection{Diffusion tensor imaging}
In addition to the T1 and T2 weighted sequences described above we have also obtained diffusion tensor imaging to assess the apparent diffusion 
coefficients on short time-scales. Images are shown in Figure~\ref{figuredti} where the largest diffusion coefficient (shown in 
red in the middle figure) is shown to be around $1.0\mathrm{e}{-3}  \mathrm{mm^2/s}$. We remark that we have not included possible anisotropy, shown in 
the right-most image in Figure~\ref{figuredti} and that these images show the apparent diffusion coefficient for free water molecules (18 Da).      
The diffusivity of the Gadovist (600 Da) \cite{MGadobutrol} was estimated to be similar to the diffusion coefficient of Gd-DPTA (550 Da) \cite{MGgDPTA}. This is due to the fact that both molecules have similar mass, and based on Stoke-Einstein equation should also have similar diffusion coefficients. The free diffusion coefficient for Gd-DPTA was estimated in \citet{GdDPTA-DIFFUSION} to be $3.8\mathrm{e}{-4}\mathrm{mm\sp{2}/s}$.
The fractional anisotropy is defined as 
\begin{equation}
FA\sp{2} =  \frac{3}{2} \frac{ (\lambda_1 - MD )\sp{2} +(\lambda_2 - MD )\sp{2} +(\lambda_3 - MD )\sp{2}}{\lambda\sp{2}_1 + \lambda\sp{2}_2  +\lambda\sp{2}_3 },
\end{equation}
with the mean diffusivity $MD$ defined as 
\begin{equation}
MD = \frac{\lambda_1 +\lambda_2 +\lambda_3 }{3}.
\end{equation}
In these equations $\lambda_i$ denotes the eigenvalues of the diffusion tensor.
\begin{figure}
\centering
\includegraphics[width=0.75\textwidth]{../DTI-zoom.png} 
\caption{The left panel shows the anatomical map. The middle panel shows the apparent diffusion coefficients (ADC) obtained from DTI. The right panel shows the computed fractional anisotropy (FA) from the DTI.}
\label{figuredti} 
\end{figure}



\begin{table}
\centering
\caption{Shows the regularization parameters $\alpha$ and $\beta$, number of timesteps $k$ with the resulting number of optimization iterations and the relative error for the control parameters. The observations times were chosen $ t_i=[2.4, 4.8,$ 7.2, 9.6, 12.0, 14.4, $16.8, 19.2, 21.6, 24.0]$.}
\begin{tabular}{*{8}c}
$\alpha$ & $\beta$ & k  & iter & $ D_{\Omega_1}\sp{rel}$ & $ D_{\Omega_2}\sp{rel} $ & $D_{\Omega_3}\sp{rel}$ & $||g||^{rel}$ \\
\hline 
 1.0e+00 	 & 1.0e+02 	 & 10 & 437 	 & +4.631 & +0.283 & +0.103 & +0.127 \\ 
 1.0e+00 	 & 1.0e+00 	 & 10 & 129 	 & +1.638 & +0.224 & +0.089 & +0.119 \\ 
 1.0e+00 	 & 1.0e-02 	 & 10 & 129 	 & +1.602 & +0.223 & +0.089 & +0.119 \\ 
 1.0e+00 	 & 1.0e-04 	 & 10 & 151 	 & +1.608 & +0.223 & +0.089 & +0.119 \\ 
 1.0e+00 	 & 1.0e-06 	 & 10 & 133 	 & +1.603 & +0.223 & +0.089 & +0.119 \\ 

 1.0e-02 	 & 1.0e+02 	 & 10 & 186 	 & +0.649 & +0.111 & +0.036 & +0.033 \\ 
 1.0e-02 	 & 1.0e+00 	 & 10 & 125 	 & -0.027 & +0.041 & +0.036 & +0.026 \\ 
 1.0e-02 	 & 1.0e-02 	 & 10 & 133 	 & -0.039 & +0.039 & +0.036 & +0.026 \\ 
 1.0e-02 	 & 1.0e-04 	 & 10 & 108 	 & -0.039 & +0.039 & +0.036 & +0.026 \\ 
 1.0e-02 	 & 1.0e-06 	 & 10 & 154 	 & -0.039 & +0.039 & +0.036 & +0.026 \\ 

 1.0e-04 	 & 1.0e+02 	 & 10 & 212 	 & +0.636 & +0.108 & +0.035 & +0.032 \\ 
 1.0e-04 	 & 1.0e+00 	 & 10 & 143 	 & -0.036 & +0.039 & +0.035 & +0.026 \\ 
 1.0e-04 	 & 1.0e-02 	 & 10 & 108 	 & -0.047 & +0.037 & +0.036 & +0.026 \\ 
 1.0e-04 	 & 1.0e-04 	 & 10 & 158 	 & -0.047 & +0.037 & +0.036 & +0.026 \\ 
 1.0e-04 	 & 1.0e-06 	 & 10 & 142 	 & -0.047 & +0.037 & +0.036 & +0.026 \\ 

 1.0e-06 	 & 1.0e+02 	 & 10 & 203 	 & +0.636 & +0.108 & +0.035 & +0.032 \\ 
 1.0e-06 	 & 1.0e+00 	 & 10 & 117 	 & -0.035 & +0.039 & +0.035 & +0.026 \\ 
 1.0e-06 	 & 1.0e-02 	 & 10 & 120 	 & -0.047 & +0.037 & +0.036 & +0.026 \\ 
 1.0e-06 	 & 1.0e-04 	 & 10 & 122 	 & -0.047 & +0.037 & +0.036 & +0.026 \\ 
 1.0e-06 	 & 1.0e-06 	 & 10 & 133 	 & -0.047 & +0.037 & +0.036 & +0.026 \\ 

\end{tabular}
\label{Tab::1}
\end{table} 
 

\begin{table}[t]
\centering
\caption{Shows the relaxation parameters $\alpha$ and $\beta$, number of timesteps $k$, the resulting number of iterations, the relative error of the estimated optimal parameters for the diffusion coefficients and the relative error for $g$. The noise amplitude was set to 0.3, and $t_i =[2.4, 4.8, 7.2, 9.6, 12.0, 14.4, 16.8, 19.2, 21.6, 24.0]$  }
\begin{tabular}{*{8}c}
$\alpha$ & $\beta$ & k  & iter & $ D_{\Omega_1}\sp{rel}$ & $D_{\Omega_2}\sp{rel} $ & $D_{\Omega_3}\sp{rel} $ & $||g||\sp{rel} $ \\
\hline
 1.0e-02 	 & 1.0e+00 	 & 10 & 144 	 & +0.012 & +0.055 & +0.027 & +0.036 \\ 
 1.0e-02 	 & 1.0e-02 	 & 10 & 476 	 & -0.009 & +0.049 & +0.035 & +0.051 \\ 
 1.0e-02 	 & 1.0e-04 	 & 10 & 476 	 & -0.097 & +0.042 & +0.054 & +0.054 \\ 
 1.0e-02 	 & 1.0e-06 	 & 10 & 558 	 & -0.071 & +0.047 & +0.039 & +0.054 \\ 
 1.0e-04 	 & 1.0e+00 	 & 10 & 182 	 & -0.040 & +0.033 & +0.051 & +0.037 \\ 
 1.0e-04 	 & 1.0e-02 	 & 10 & 471 	 & +0.021 & +0.038 & +0.036 & +0.056 \\ 
 1.0e-04 	 & 1.0e-04 	 & 10 & 792 	 & -0.058 & +0.042 & +0.031 & +0.565 \\ 
 1.0e-04 	 & 1.0e-06 	 & 10 &  -   & -0.050 & +0.029 & +0.040 & +1.062 \\ 
 1.0e-06 	 & 1.0e+00 	 & 10 & 178 	 & +0.012 & +0.033 & +0.021 & +0.036 \\ 
 1.0e-06 	 & 1.0e-02 	 & 10 & 537 	 & -0.018 & +0.040 & +0.025 & +0.057 \\ 
 1.0e-06 	 & 1.0e-04 	 & 10 &  -	 & -0.038 & +0.040 & +0.030 & +1.452 \\ 
 1.0e-06 	 & 1.0e-06 	 & 10 &  - 	 & +0.003 & +0.035 & +0.046 & +3.621 \\ 
 1.0e-02 	 & 1.0e+00 	 & 20 & 213 	 & +0.009 & +0.015 & +0.009 & +0.016 \\ 
 1.0e-02 	 & 1.0e-02 	 & 20 & 719 	 & +0.228 & +0.088 & -0.032 & +0.085 \\ 
 1.0e-02 	 & 1.0e-04 	 & 20 &  -   & +0.100 & +0.071 & -0.013 & +0.271 \\ 
 1.0e-02 	 & 1.0e-06 	 & 20 &   -  & +0.223 & +0.097 & -0.027 & +0.282 \\ 
 1.0e-04 	 & 1.0e+00 	 & 20 & 245 	 & +0.038 & +0.013 & -0.001 & +0.016 \\ 
 1.0e-04 	 & 1.0e-02 	 & 20 &  - 	 & +0.050 & +0.054 & -0.024 & +0.116 \\ 
 1.0e-04 	 & 1.0e-04 	 & 20 &  - 	 & -0.087 & +0.059 & -0.077 & +5.763 \\ 
 1.0e-04 	 & 1.0e-06 	 & 20 &  -	 & +0.001 & +0.083 & -0.038 & +7.485 \\ 
 1.0e-06 	 & 1.0e+00 	 & 20 & 242 	 & -0.009 & +0.008 & +0.015 & +0.016 \\ 
 1.0e-06 	 & 1.0e-02 	 & 20 & 947 	 & +0.122 & +0.093 & -0.040 & +0.115 \\ 
 1.0e-06 	 & 1.0e-04 	 & 20 &  - 	 & -0.076 & +0.041 & -0.020 & +7.700 \\ 
 1.0e-06 	 & 1.0e-06 	 & 20 &  - 	 & +0.147 & +0.113 & -0.023 & +10.936 \\ 

\end{tabular}
\label{Tab::Noise0.3}
\end{table} 


\begin{table}
\centering
\caption{Shows the relaxation parameters $\alpha$ and $\beta$, number of timesteps $k$, the resulting number of iterations, the estimated diffusion coefficients for grey $D_{\Omega_{GM}}$ and white $D_{\Omega_{WM}}$ matter based on MRI data.}
\begin{tabular}{*{6}c}
$\alpha$ & $\beta$ & k & iter &  $ D_{\Omega_{GM}} \left[ \mathrm{mm\sp{2}/h} \right] $ & $ D_{\Omega_{WM}} \left[ \mathrm{mm\sp{2}/h} \right]$ \\
\hline
 1.0e-02 	 & 1.0e+00 	 & 24 & 503 	 & 0.659 & 0.918 \\ 
 1.0e-02 	 & 1.0e-01 	 & 24 & 716 	 & 0.771 & 0.969 \\ 
 1.0e-02 	 & 1.0e-02 	 & 24 & 505 	 & 0.836 & 0.974 \\ 
 1.0e-04 	 & 1.0e+00 	 & 24 & 206 	 & 0.553 & 0.828 \\ 
 1.0e-04 	 & 1.0e-01 	 & 24 & 440 	 & 0.651 & 0.789 \\ 
 1.0e-04 	 & 1.0e-02 	 & 24 & 597 	 & 0.718 & 0.739 \\ 
 1.0e-06 	 & 1.0e+00 	 & 24 & 439 	 & 0.570 & 0.830 \\ 
 1.0e-06 	 & 1.0e-01 	 & 24 & 583 	 & 0.656 & 0.793 \\ 
 1.0e-06 	 & 1.0e-02 	 & 24 & 742 	 & 0.721 & 0.729 \\ 
 
 %1.0e-02    & 1.0e+02   & 48 & 714      & 0.564 & 0.952 ?? \\ 
 1.0e-02 	 & 1.0e+00 	 & 48 & 591 	 & 0.648 & 0.893 \\ 
 1.0e-02 	 & 1.0e-01 	 & 48 & 461 	 & 0.772 & 0.982 \\ 
 1.0e-02 	 & 1.0e-02 	 & 48 & 748 	 & 0.986 & 1.198 \\ 
 %1.0e-04    & 1.0e+02   & 48 & 815      & 0.470 & 0.928 \\ 
 1.0e-04 	 & 1.0e+00 	 & 48 & 607 	 & 0.551 & 0.815 \\ 
 1.0e-04 	 & 1.0e-01 	 & 48 & 598 	 & 0.649 & 0.787 \\ 
 1.0e-04 	 & 1.0e-02 	 & 48 & 837 	 & 0.740 & 0.817 \\
 %1.0e-06    & 1.0e+02   & 48 & 729      & 0.428 & 0.948 \\ 
 1.0e-06 	 & 1.0e+00 	 & 48 & 681 	 & 0.557 & 0.811 \\ 
 1.0e-06 	 & 1.0e-01 	 & 48 & 780 	 & 0.645 & 0.774 \\ 
 1.0e-06 	 & 1.0e-02 	 & 48 & 745 	 & 0.736 & 0.819 \\ 



\end{tabular}
\label{Tab::Real-data}
\end{table} 
 



\subsubsection{The number of timessteps}
Increasing the number of time-steps in the computations can potentially allow for a more accurate temporal reconstruction, but at the same time the number of observations per control decreases and the regularization may therefore play a larger role. Furthermore, the computational cost will rise with additional time-steps. Thus finding the optimal number of time-steps can be essential for solving larger problems. The computation used a wide range the regularization parameters, and the number of timesteps were selected to be 20 and 40.

The results are given in Table~\ref{TAB::timesteps}. It can be seen for all $\alpha=1.0$ that the relative error becomes bigger with more time-steps. This behavior can also be seen for $\alpha=1.0\mathrm{e}{-2}$ and $\beta < 1.0\mathrm{e}{-2}$, which implies an interaction between the regularization parameters. Therefore $g$ was investigated by plotting boundary points through time for different values of $\alpha$ and $\beta$, see Fig \ref{boundarycontrol}.

\begin{table}
\centering
\caption{ Shows the relaxation parameters $\alpha$ and $\beta$, number of timesteps $k$, the resulting number of iterations, the relative error of the estimated diffusion coefficients and the relative error for $g$. The observations times were chosen $ t_i=[2.4,  4.8,$ 7.2, 9.6, 12.0, 14.4, $16.8, 19.2, 21.6, 24.0]$ }
\begin{tabular}{*{8}c}
$\alpha$ & $\beta$ & k & iter & $ D_{\Omega_1}\sp{rel}$& $D_{\Omega_2}\sp{rel} $ & $D_{\Omega_3}\sp{rel} $&$|| g ||\sp{rel} $ \\
\hline
 1.0e+00 	 & 1.0e+02 	 & 20 & 600 	 & +6.412 & +0.175 & +0.086 & +0.124 \\
 1.0e+00 	 & 1.0e+02 	 & 40 &  -   & +6.702 & +0.130 & +0.082 & +0.125 \\  
 1.0e+00 	 & 1.0e+00 	 & 20 & 300 	 & +5.538 & +0.324 & +0.148 & +0.128 \\ 
 1.0e+00 	 & 1.0e+00 	 & 40 & 346 	 & +11.392 & +0.873 & +0.242 & +0.154 \\ 
 1.0e+00 	 & 1.0e-02 	 & 20 & 415 	 & +18.156 & +1.378  & +0.389  & +0.251 \\ 
 1.0e+00 	 & 1.0e-02 	 & 40 & 856 	 & +64.050 & +17.408 & +15.950 & +0.614 \\ 
 1.0e+00 	 & 1.0e-04 	 & 20 & 417 	 & +17.747 & +1.465  & +0.406  & +0.258 \\ 
 1.0e+00 	 & 1.0e-04 	 & 40 & 946 	 & +72.594 & +17.985 & +16.702 & +0.641 \\
 1.0e+00 	 & 1.0e-06 	 & 20 & 399 	 & +17.407 & +1.466  & +0.406  & +0.258 \\ 
 1.0e+00 	 & 1.0e-06 	 & 40 & 863 	 & +60.448 & +18.043 & +16.732 & +0.641 \\
 \hline
 1.0e-02 	 & 1.0e+02 	 & 20 & 351 	 & +0.865 & +0.001 & +0.017 & +0.027 \\ 
 1.0e-02 	 & 1.0e+02 	 & 40 & 404 	 & +0.846 & -0.049 & +0.010 & +0.026 \\ 
 1.0e-02 	 & 1.0e+00 	 & 20 & 254 	 & +0.018 & +0.007 & +0.008 & +0.007 \\ 
 1.0e-02 	 & 1.0e+00 	 & 40 & 218 	 & +0.020 & -0.006 & +0.001 & +0.003 \\ 
 1.0e-02 	 & 1.0e-02 	 & 20 & 381 	 & +0.127 & +0.057 & -0.001 & +0.016 \\ 
 1.0e-02 	 & 1.0e-02 	 & 40 & 543 	 & +0.099 & +0.087 & +0.002 & +0.023 \\
 1.0e-02 	 & 1.0e-04 	 & 20 & 641 	 & +0.171 & +0.082 & -0.003 & +0.091 \\ 
 1.0e-02 	 & 1.0e-04 	 & 40 & 879 	 & +0.303 & +0.383 & +0.079 & +0.222 \\ 
 1.0e-02 	 & 1.0e-06 	 & 20 & 547 	 & +0.173 & +0.084 & -0.003 & +0.092 \\
 1.0e-02 	 & 1.0e-06 	 & 40 & 844 	 & +0.332 & +0.416 & +0.095 & +0.239 \\ 
 \hline 
 1.0e-04 	 & 1.0e+02 	 & 20 & 257 	 & +0.853 & -0.001 & +0.017 & +0.026 \\ 
 1.0e-04 	 & 1.0e+02 	 & 40 & 532 	 & +0.822 & -0.051 & +0.010 & +0.026 \\ 
 1.0e-04 	 & 1.0e+00 	 & 20 & 164 	 & +0.008 & +0.003 & +0.008 & +0.007 \\ 
 1.0e-04 	 & 1.0e+00 	 & 40 & 203 	 & +0.005 & -0.012 & +0.001 & +0.003 \\  
 1.0e-04 	 & 1.0e-02 	 & 20 & 313 	 & +0.071 & +0.021 & -0.001 & +0.007 \\ 
 1.0e-04 	 & 1.0e-02 	 & 40 & 294 	 & +0.004 & +0.004 & -0.001 & +0.001 \\ 
 1.0e-04 	 & 1.0e-04 	 & 20 & 265 	 & +0.108 & +0.035 & -0.002 & +0.014 \\
 1.0e-04 	 & 1.0e-04 	 & 40 & 401 	 & -0.010 & -0.002 & -0.002 & +0.004 \\ 
 1.0e-04 	 & 1.0e-06 	 & 20 & 452 	 & +0.066 & +0.021 & -0.003 & +0.031 \\ 
 1.0e-04 	 & 1.0e-06 	 & 40 & 330 	 & -0.014 & -0.013 & -0.003 & +0.004 \\ 
 \hline
 1.0e-06 	 & 1.0e+02 	 & 20 & 274 	 & +0.850 & -0.001 & +0.017 & +0.026 \\ 
 1.0e-06 	 & 1.0e+02 	 & 40 & 496 	 & +0.821 & -0.051 & +0.010 & +0.026 \\
 1.0e-06 	 & 1.0e+00 	 & 20 & 176 	 & +0.008 & +0.003 & +0.008 & +0.007 \\ 
 1.0e-06 	 & 1.0e+00 	 & 40 & 207 	 & +0.006 & -0.012 & +0.001 & +0.003 \\ 
 1.0e-06 	 & 1.0e-02 	 & 20 & 223 	 & +0.075 & +0.024 & -0.001 & +0.007 \\ 
 1.0e-06 	 & 1.0e-02 	 & 40 & 392 	 & +0.000 & -0.001 & -0.001 & +0.002 \\ 
 1.0e-06 	 & 1.0e-04 	 & 20 & 429 	 & +0.085 & +0.027 & -0.002 & +0.025 \\ 
 1.0e-06 	 & 1.0e-04 	 & 40 & 241 	 & -0.020 & -0.030 & -0.005 & +0.005 \\ 
 1.0e-06 	 & 1.0e-06 	 & 20 & 591 	 & +0.060 & +0.021 & -0.003 & +0.048 \\ 
 1.0e-06 	 & 1.0e-06 	 & 40 & 343 	 & -0.014 & -0.010 & -0.003 & +0.004 \\ 

\end{tabular}
\label{TAB::timesteps}
\end{table} 



\subsubsection{The number of observations}
The dependency on observations is relevant, since the number of observations of the MRI data is limited. This dependency was investigated by changing the number of observations, and by examining the results. The number of observation were chosen to be 5 and 20, and the observation times were evenly spaced. 

The results for 5 observations is shown in Table~\ref{TAB::half} and for 20 observations is shown in Table~\ref{TAB::double}. In Table~\ref{TAB::half}, the values $\alpha =1.0\mathrm{e}{-2}$ and $\beta<1.0$ gives a surge in the relative error with more time-steps. This behavior is neither visible in Table~\ref{TAB::timesteps} or in  Table~\ref{TAB::double} for the same regularization parameters. The ratio observations per control indicates that the increase in relative error is due to oscillations, similar to those seen in Figure~\ref{boundarycontrol}. Thus the observations acts as stabilization for the functional, which gives a broader range of adequate regularization parameters.


\begin{table}
\centering
\caption{ Shows the relaxation parameters $\alpha$ and $\beta$, number of timesteps $k$, the resulting number of iterations, the relative error of the estimated diffusion coefficients and the relative error for $g$. The observation times were set $t_i = [ 4.8, 9.6, 14.4, 19.2, 24.0 ] $. }
\begin{tabular}{*{8}c}
$\alpha$ & $\beta$ & k & iter & $ D_{\Omega_1}\sp{rel}$& $D_{\Omega_2}\sp{rel} $ & $D_{\Omega_3}\sp{rel} $& $||g||\sp{rel} $ \\
\hline
 1.0e-02 	 & 1.0e+00 	 & 10 & 151 	 & +0.095 & +0.015 & +0.028 & +0.020 \\ 
 1.0e-02 	 & 1.0e+00 	 & 20 & 165 	 & +0.092 & -0.001 & -0.001 & +0.008 \\ 
 1.0e-02 	 & 1.0e+00 	 & 40 & 228 	 & +0.067 & -0.024 & -0.011 & +0.005 \\ 
 
 1.0e-02 	 & 1.0e-02 	 & 10 & 330 	 & +0.190 & +0.074 & +0.027 & +0.083 \\ 
 1.0e-02 	 & 1.0e-02 	 & 20 & 455 	 & +3.882 & +3.297 & +2.583 & +0.304 \\ 
  1.0e-02 	 & 1.0e-02 	 & 40 & 523 	 & +7.466 & +6.454 & +5.537 & +0.388 \\ 

 1.0e-02 	 & 1.0e-04 	 & 10 & 428 	 & +0.196 & +0.076 & +0.032 & +0.163 \\ 
 1.0e-02 	 & 1.0e-04 	 & 20 & 571 	 & +7.402 & +6.421 & +5.405 & +0.680 \\ 
 1.0e-02 	 & 1.0e-04 	 & 40 & 747 	 & +15.386 & +13.764 & +12.196 & +0.841 \\ 
  
  1.0e-04 	 & 1.0e+00 	 & 10 & 147 	 & +0.081 & +0.008 & +0.023 & +0.020 \\ 
  1.0e-04 	 & 1.0e+00 	 & 20 & 167 	 & +0.051 & -0.026 & -0.009 & +0.008 \\ 
  1.0e-04 	 & 1.0e+00 	 & 40 & 148 	 & +0.018 & -0.057 & -0.020 & +0.005 \\ 
  
 1.0e-04 	 & 1.0e-02 	 & 10 & 255 	 & +0.171 & +0.087 & +0.010 & +0.021 \\ 
 1.0e-04 	 & 1.0e-02 	 & 20 & 216 	 & -0.005 & -0.006 & -0.008 & +0.008 \\ 
 1.0e-04 	 & 1.0e-02 	 & 40 & 386 	 & -0.004 & -0.012 & -0.015 & +0.004 \\ 
  
  
 1.0e-04 	 & 1.0e-04 	 & 10 & 284 	 & +0.059 & +0.052 & +0.010 & +0.051 \\ 
 1.0e-04 	 & 1.0e-04 	 & 20 & 256 	 & -0.048 & -0.045 & -0.024 & +0.020 \\ 
  1.0e-04 	 & 1.0e-04 	 & 40 & 277 	 & -0.071 & -0.074 & -0.050 & +0.018 \\ 
  
 1.0e-06 	 & 1.0e+00 	 & 10 & 163 	 & +0.082 & +0.008 & +0.023 & +0.020 \\ 
 1.0e-06 	 & 1.0e+00 	 & 20 & 156 	 & +0.054 & -0.027 & -0.009 & +0.008 \\ 
 1.0e-06 	 & 1.0e+00 	 & 40 & 284 	 & +0.020 & -0.055 & -0.019 & +0.005 \\
  
 1.0e-06 	 & 1.0e-02 	 & 10 & 290 	 & +0.155 & +0.075 & +0.011 & +0.021 \\ 
 1.0e-06 	 & 1.0e-02 	 & 20 & 239 	 & -0.004 & +0.002 & -0.006 & +0.008 \\ 
 1.0e-06 	 & 1.0e-02 	 & 40 & 378 	 & -0.005 & -0.013 & -0.016 & +0.004 \\ 
  
 1.0e-06 	 & 1.0e-04 	 & 10 & 264 	 & +0.070 & +0.056 & +0.009 & +0.050 \\ 
 1.0e-06 	 & 1.0e-04 	 & 20 & 234 	 & -0.047 & -0.044 & -0.029 & +0.020 \\
 1.0e-06 	 & 1.0e-04 	 & 40 & 324 	 & -0.070 & -0.070 & -0.052 & +0.017 \\ 

\end{tabular}
\label{TAB::half}
\end{table} 

\begin{table}
\centering
\caption{ Shows the relaxation parameters $\alpha$ and $\beta$, number of timesteps $k$, the resulting number of iterations, the relative error of the estimated diffusion coefficients and the relative error for $g$. The observation times were $t_i = [ $1.2, 2.4, 3.6, 4.8, 6.0, 7.2, 8.4, 9.6, 10.8, 12.0, 13.2, 14.4, 15.6, 16.8, 17.0, 19.2, 20.4,$ 21.6, 22.8, 24.0 ]$.}
\begin{tabular}{*{8}c}
$\alpha$ & $\beta$ & k & iter & $ D_{\Omega_1}\sp{rel}$ & $ D_{\Omega_2}\sp{rel}$ & $D_{\Omega_3}\sp{rel} $ & $||g||\sp{rel} $ \\
\hline

 1.0e-02 	 & 1.0e+00 	 & 10 & 250 	 & -0.038 & +0.035 & +0.019 & +0.033 \\ 
 1.0e-02 	 & 1.0e+00 	 & 20 & 225 	 & -0.002 & +0.015 & +0.007 & +0.007 \\ 
 1.0e-02 	 & 1.0e+00 	 & 40 & 255 	 & +0.003 & +0.002 & +0.001 & +0.002 \\ 

 1.0e-02 	 & 1.0e-02 	 & 10 & 180 	 & -0.042 & +0.034 & +0.020 & +0.033 \\ 
 1.0e-02 	 & 1.0e-02 	 & 20 & 445 	 & +0.010 & +0.017 & +0.003 & +0.010 \\ 
  1.0e-02 	 & 1.0e-02 	 & 40 & 429 	 & +0.037 & +0.014 & -0.003 & +0.004 \\ 
  
 1.0e-02 	 & 1.0e-04 	 & 10 & 208 	 & -0.042 & +0.034 & +0.020 & +0.033 \\ 
  1.0e-02 	 & 1.0e-04 	 & 20 & 530 	 & +0.012 & +0.017 & +0.003 & +0.020 \\ 
 1.0e-02 	 & 1.0e-04 	 & 40 & - 	 & +0.064 & +0.023 & -0.005 & +0.072 \\ 
 
 1.0e-04 	 & 1.0e+00 	 & 10 & 209 	 & -0.043 & +0.033 & +0.019 & +0.033 \\ 
 1.0e-04 	 & 1.0e+00 	 & 20 & 212 	 & -0.007 & +0.014 & +0.007 & +0.007 \\ 
 1.0e-04 	 & 1.0e+00 	 & 40 & 258 	 & -0.002 & +0.001 & +0.001 & +0.002 \\ 
 
 1.0e-04 	 & 1.0e-02 	 & 10 & 258 	 & -0.047 & +0.033 & +0.020 & +0.033 \\ 
 1.0e-04 	 & 1.0e-02 	 & 20 & 391 	 & +0.005 & +0.016 & +0.003 & +0.008 \\ 
 1.0e-04 	 & 1.0e-02 	 & 40 & 261 	 & +0.014 & +0.005 & -0.002 & +0.001 \\ 
 
 1.0e-04 	 & 1.0e-04 	 & 10 & 267 	 & -0.047 & +0.033 & +0.020 & +0.033 \\ 
 1.0e-04 	 & 1.0e-04 	 & 20 & 769 	 & +0.008 & +0.016 & +0.003 & +0.016 \\
 1.0e-04 	 & 1.0e-04 	 & 40 & 375 	 & +0.027 & +0.005 & -0.003 & +0.002 \\ 
 
 1.0e-06 	 & 1.0e+00 	 & 10 & 198 	 & -0.042 & +0.033 & +0.019 & +0.033 \\ 
 1.0e-06 	 & 1.0e+00 	 & 20 & 222 	 & -0.007 & +0.014 & +0.007 & +0.007 \\ 
 1.0e-06 	 & 1.0e+00 	 & 40 & 280 	 & -0.001 & +0.001 & +0.001 & +0.002 \\ 
 
 1.0e-06 	 & 1.0e-02 	 & 10 & 279 	 & -0.047 & +0.033 & +0.020 & +0.033 \\ 
 1.0e-06 	 & 1.0e-02 	 & 20 & 312 	 & +0.005 & +0.016 & +0.003 & +0.008 \\ 
 1.0e-06 	 & 1.0e-02 	 & 40 & 379 	 & +0.015 & +0.005 & -0.002 & +0.001 \\
 
 1.0e-06 	 & 1.0e-04 	 & 10 & 251 	 & -0.047 & +0.033 & +0.020 & +0.033 \\ 
 1.0e-06 	 & 1.0e-04 	 & 20 & 778 	 & +0.008 & +0.016 & +0.002 & +0.018 \\ 
 1.0e-06 	 & 1.0e-04 	 & 40 & 191 	 & +0.017 & +0.004 & -0.002 & +0.002 \\ 
 

\end{tabular}
\label{TAB::double}
\end{table} 



In Table~\ref{Tab::Noise0.03}, we can see that a noise amplitude of 0.03, (10$\%$ of maximum initial condition) had negligible effect on the relative errors compared zero noise in Table~\ref{Tab::1} and Table~\ref{TAB::timesteps}. 

\begin{table}
\centering
\caption{Shows the relaxation parameters $\alpha$ and $\beta$, number of timesteps $k$, the resulting number of iterations, the relative error of the estimated optimal parameters for the diffusion coefficients and the relative error for $g$. The noise amplitude was set to 0.03, and $ t_i=[2.4, 4.8, 7.2, 9.6, 12.0, 14.4, 16.8, 19.2, 21.6, 24.0]$ }
\begin{tabular}{*{8}c}
$\alpha$ & $\beta$ & k  & iter & $ D_{\Omega_1}\sp{rel} $ & $D_{\Omega_2}\sp{rel}$ & $D_{\Omega_3}\sp{rel}$ & $||g||\sp{rel} $\\
\hline
 1.0e-02 	 & 1.0e+00 	 & 10 & 151 	 & -0.030 & +0.041 & +0.036 & +0.026 \\ 
 1.0e-02 	 & 1.0e-02 	 & 10 & 120 	 & -0.032 & +0.039 & +0.036 & +0.026 \\ 
 1.0e-02 	 & 1.0e-04 	 & 10 & 125 	 & -0.038 & +0.039 & +0.037 & +0.026 \\ 
 1.0e-02 	 & 1.0e-06 	 & 10 & 167 	 & -0.040 & +0.039 & +0.036 & +0.026 \\ 
 1.0e-04 	 & 1.0e+00 	 & 10 & 161 	 & -0.038 & +0.039 & +0.035 & +0.026 \\ 
 1.0e-04 	 & 1.0e-02 	 & 10 & 95 	 & -0.047 & +0.036 & +0.036 & +0.026 \\ 
 1.0e-04 	 & 1.0e-04 	 & 10 & 75 	 & -0.041 & +0.036 & +0.035 & +0.026 \\ 
 1.0e-04 	 & 1.0e-06 	 & 10 & 136 	 & -0.044 & +0.036 & +0.036 & +0.026 \\ 
 1.0e-06 	 & 1.0e+00 	 & 10 & 134 	 & -0.034 & +0.037 & +0.037 & +0.026 \\ 
 1.0e-06 	 & 1.0e-02 	 & 10 & 112 	 & -0.046 & +0.038 & +0.036 & +0.026 \\ 
 1.0e-06 	 & 1.0e-04 	 & 10 & 111 	 & -0.043 & +0.036 & +0.037 & +0.026 \\ 
 1.0e-06 	 & 1.0e-06 	 & 10 & 214 	 & -0.049 & +0.037 & +0.036 & +0.026 \\
  
 1.0e-02 	 & 1.0e+00 	 & 20 & 190 	 & +0.013 & +0.006 & +0.008 & +0.007 \\ 
 1.0e-02 	 & 1.0e-02 	 & 20 & 392 	 & +0.126 & +0.057 & +0.001 & +0.017 \\ 
 1.0e-02 	 & 1.0e-04 	 & 20 & 677 	 & +0.164 & +0.074 & -0.001 & +0.093 \\ 
 1.0e-02 	 & 1.0e-06 	 & 20 & 600 	 & +0.179 & +0.081 & -0.001 & +0.098 \\ 
 1.0e-04 	 & 1.0e+00 	 & 20 & 224 	 & +0.006 & +0.004 & +0.008 & +0.007 \\ 
 1.0e-04 	 & 1.0e-02 	 & 20 & 465 	 & +0.074 & +0.022 & -0.003 & +0.008 \\ 
 1.0e-04 	 & 1.0e-04 	 & 20 & 586 	 & +0.071 & +0.023 & -0.005 & +0.056 \\ 
 1.0e-04 	 & 1.0e-06 	 & 20 & 472 	 & +0.081 & +0.024 & -0.003 & +0.056 \\ 
 1.0e-06 	 & 1.0e+00 	 & 20 & 157 	 & +0.011 & +0.003 & +0.009 & +0.007 \\ 
 1.0e-06 	 & 1.0e-02 	 & 20 & 352 	 & +0.076 & +0.021 & +0.000 & +0.008 \\ 
 1.0e-06 	 & 1.0e-04 	 & 20 & 738 	 & +0.083 & +0.026 & -0.004 & +0.065 \\ 
 1.0e-06 	 & 1.0e-06 	 & 20 & 599 	 & +0.056 & +0.021 & -0.003 & +0.106 \\ 
 

\end{tabular}
\label{Tab::Noise0.03}
\end{table} 


